Русслая Прафпсмафоая Верлпфь
Ппсмедпфаойе Бпкестфеоопй Лйтурзйй
Nikolas Polkin 23.06.2010

мйтурзйя пзмахеооци
Благослови, владыко.
Благословенно Царство Отца и Сына и Святаго
Духа ныне и присно и во веки веков.
Диакон:
Иерей:
Хор: Аминь.
Фемйлая Елтеойя
Миром Господу помолимся.
Хор: (На каждое прошение)
О Свышнем мире и спасении душ наших Господу помолимся.
О мире всего мира, благостоянии Святых Божиих Церквей и соединении всех Господу помолимся.
О святем храме сем и с верою, благоговением и страхом Божиим входящих в онь Господу помолимся.
О Великом Господине и Отце нашем Святейшем Патри- арсе Кирилле, и о Господине нашем Преосвященнейшем (имя епархиального архиерея), честнем пресвитерстве, во Христе диаконстве, о всем причте и людех Господу помо- лимся.
О Богохранимей стране нашей, властех и воинстве ея Господу помолимся.
О граде сем (или: о веси сей; если в монастыре, то: о святой обители сей), всяком граде, стране и верою живущих в них Господу помолимся.
О благорастворении воздухов, о изобилии плодов земных и временех мирных Господу помолимся.
Диакон:
Господи помилуй.
Диакон:
~2~

О плавающих, путешествующих, недугующих, страждущих, плененных и о спасении их Господу помолимся.
О избавитися нам от всякия скорби, гнева и нужды Господу помолимся.
Заступи, спаси, помилуй и сохрани нас, Боже, Твоею благодатию.
Пресвятую, Пречистую, Преблагословенную, Славную Владычицу нашу Богородицу и Приснодеву Марию, со все- ми святыми помянувше, сами себе и друг друга, и весь живот наш Христу Богу предадим.
Хор: Тебе, Господи.
Тайная молитва первого антифона
Иерей: Господи Боже наш, Егоже держава несказанна и слава непостижима, Егоже милость безмерна и человеколюбие неизреченно! Сам, Владыко, по благоутробию Твоему, призри на ны и на святый храм
сей и сотвори с нами и молящимися с нами богатыя милости Твоя и щедроты Твоя.
Иерей: Яко подобает Тебе всякая слава, честь и поклоне- ние, Отцу и Сыну и Святому Духу, ныне и присно и во веки веков.
Хор: Аминь.
Аотйжпоц йчпбрачйтемьоце:
Перфцй аотйжпо
Благослови, душе моя, Господа./ Благословен еси, Господи./ Благослови, душе моя, Господа,/ и вся внутренняя моя Имя Святое Его./
Благослови, душе моя, Господа,/ и не забывай всех воздаяний Его./
~3~

Очищающаго вся беззакония твоя,/ исцеляющаго вся недуги твоя./
Творяй милостыни Господь,/ и судьбу всем обидимым./ Щедр и милостив Господь,/ долготерпелив и многомилстив./
Благослови, душе моя, Господа,/ и вся внутреняя моя,/ Имя Святое Его.// Благословен еси, Господи.
Мамая елтеойя
Паки и паки миром Господу помолимся. Хор: (На оба прошения)
Заступи, спаси, помилуй и сохрани нас, Боже, Твоею благодатию.
Пресвятую, Пречистую, Преблагословенную, Славную Владычицу нашу Богородицу и Приснодеву Марию, со всеми святыми помянувше, сами себе и друг друга, и весь живот наш Христу Богу предадим.
Хор: Тебе, Господи.
Тайная молитва второго антифона
Иерей: Господи Боже наш! Спаси люди Твоя и благослови достояние Твое, исполнение Церкве Твоея сохрани, освяти любящия благолепие дому Твоего; Ты тех воспрослави Божественною Твоею силою и не остави нас, уповающих на Тя,
Иерей: Яко Твоя держава, и Твое есть Царство, и сила, и слава, Отца и Сына и Святаго Духа, ныне и присно и во веки веков.
Хор: Амиинь.
Избавляющаго от нетления живот твой,/ венчающаго тя милостию и щедротами./
Исполняющего во благих желание твое,/ обновится, яко орля, юность твоя./
Диакон:
Господи, помилуй.
Диакон:
~4~

Фтпрпй аотйжпо
Хвали, душе моя, Господа./ Восхвалю Господа в животе моем,/ пою Богу моему, дондеже есмь,/
Не надейтеся на князи, на сыны человеческия,/ в нихже несть спасения./
Изыдет дух его, и возвратится в землю свою:/ в той день погибнут вся помышления его./
Блажен, емуже Бог Иаковль помощник его,/ упование его на Господа Бога своего./
Сотворшаго небо и землю,/ море и вся, яже в них./
Хранящаго истину в век,/ творящаго суд обидимым,/ дающаго пищу алчущим./
Господь решит окованныя,/ Господь умудряет слепцы./ Господь возводит низвирженныя,/ Господь любит праведники./ Господь хранит пришельцы,/ сира и вдову приимет,/ и путь грешных погубит./
Воцарится Господь во век,// Бог твой, Сионе, в род и род.
Слава Отцу и Сыну и Святому Духу и ныне и присно и во веки веков. Аминь.
Песоь Гпсрпду Ийсусу Хрйсту
Единородный Сыне и Слове Божий, Безсмертен Сый,/ и изволивый спасения нашего ради/ воплотитися от Святыя Богородицы и Приснодевы Марии,/ непреложно вочеловечивыйся;/ распныйся же, Христе Боже, смертию смерть поправый,/ един Сый Святыя Троицы,// спрославляемый Отцу и Святому Духу, спаси нас.
~5~

Мамая елтеойя
Паки и паки миром Господу помолимся.
Хор: (На оба, непрерывно в миноре)
Заступи, спаси, помилуй и сохрани нас, Боже, Твоею благодатию.
Пресвятую, Пречистую, Преблагословенную, Славную Владычицу нашу Богородицу и Приснодеву Марию, со всеми святыми помянувше, сами себе и друг друга, и весь живот наш Христу Богу предадим.
Хор: Тебе, Господи.
Диакон:
Господи, помилуй.
Диакон:
Тайная молитва третьего антифона
Иерей: Иже общия сия и согласный даровавый нам молитвы, Иже и двема или трем, согласующимся о имени Твоем, прошения подати обещавый! Сам и ныне раб Твоих прошения к полезному исполни, подая нам и в настоящем веце познание Твоея истины и в будущем живот вечный даруя.
Иерей: Яко Благ и Человеколюбец Бог еси, и Тебе славу возсылаем, Отцу и Сыну и Святому Духу, ныне и присно и во веки веков.
Хор: Аминь. (Протяжно )
Отверзаются Царские врата.
Третйй аотйжпо Бмакеооц
Во Царствии Твоем помяни нас, Господи,/ егда приидеши во
Царствии Твоем.
Блажени нищии духом,/ яко тех есть Царство Небесное.
Блажени плачущии,/ яко тии утешатся. ~6~
На 12

На 10
Блажени кротции,/ яко тии наследят землю.
Блажени алчущии и жаждущии правды,/ яко тии насытятся.
На 8
Блажени милостивии,/ яко тии помиловани будут. Блажени чистии сердцем,/ яко тии Бога узрят.
На 6
Блажени миротворцы,/ яко тии сынове Божии нарекутся. Блажени изгнани правды ради,/ яко тех есть Царство
Небесное.
На 4
Блажени есте, егда поносят вам,/ и изженут, и рекут всяк зол глагол на вы, лжуще Мене ради.
Радуйтеся и веселитеся,/ яко мзда ваша многа на Небесех. Слава Отцу и Сыну /и Святому Духу.
И ныне и присно /и во веки веков. Аминь.
Мамцй фипд (с Ефаоземйен)
Диакон: Господу помолимся.
Молитва входа
Иерей: Владыко Господи, Боже наш, уставивый на Небесех чины и воинства Ангел и Архангел в
служение Твоея славы! Сотвори со входом нашим входу святых Ангелов быти, сослужащих нам и сословословящих Твою благость, Яко подобает Тебе всякая слава, честь и поклонение, Отцу и Сыну и Святому Духу, ныне и присно и во веки веков. Аминь.
Диакон: Премудрость, прости.
В дни Великих Господских праздников (Пасха, Рождество Христово, Богоявление, Сретение Господне, Вход
Господень в Иерусалим, Вознесение Господне, Пятидесятница и День Святого Духа, Преображение,
Диакон: Благослови, владыко, святый вход.
Иерей (благословляя): Благословен вход святых Твоих, все- гда, ныне и присно и во веки веков.
~7~

Воздвижение Креста Господня) диакон возглашает также входной стих. В этом случае Входное не поется', а сразу тропарь и кондак праздника.
Фипдопе фпслресопе
Хор: Приидите, поклонимся и припадем ко Христу. Спаси ны, Сыне Божий, воскресый из мертвых, поющия Ти: аллилуиа.
В будние дни вместо слов воскресый из мертвых поется Во святых Дивен сый.
В Богородичные двунадесятые праздники и в дни их попразднства поется Молитвами Богородицы.
В попразднство Рождества Христова: Рождейся от Девы.
Богоявления: Во Иордане крестивыйся.
Вознесения Господня: Вознесыйся во славе.
Преображения: Преобразивыйся на горе.
Воздвижения Креста Господня: Плотию распныйся.
В попразднство Пятидесятницы вместо слов Спаси ны, Сыне Божий, воскресый из мертвых поется Спаси ны, Утешителю Благий.
Трпрарй й лподалй
Во время пения тропарей и кондаков иерей тайно читает молитву Трисвятого пения.
Иерей: Боже, Святый, Иже во святых почиваяй, Иже трисвятым гласом от Серафимов воспеваемый, и от Херувимов славословимый, и от всякия Небесный Силы покланяемый; Иже от небытия во еже быти приведый всяческая, создавый человека по образу Твоему и по подобию и всяким Твоим дарованием украсйвый; даяй просящему премудрость и разум и не презираяй согрешающаго, но полагаяй на спасение покаяние; сподобивый нас, смиренных и недостойных раб Твоих, и в час сей стати пред славою Святаго Твоего Жертвенника и должное Тебе поклонение и славословие приносити! Сам, Владыко, приими и от уст нас, грешных, Трисвятую песнь и посети ны благостию Твоего, прости нам всякое согрешение, вольное же и невольное, освяти наша души и телеса и даждь нам в преподобии служити Тебе вся дни живота нашего, молитвами Святыя Богородицы и всех святых, от века Тебе благоугодивших.
Диакон: Господу помолимся.
Хор: Господи, помилуй.
Иерей: Яко Свят еси, Боже наш, и Тебе славу возсылаем, Отцу и Сыну и Святому Духу, ныне и присно.
Диакон: Господи, спаси благочестивыя.
Хор: Господи, спаси благочестивыя.
Диакон: И услыши ны.
Хор: И услыши ны.
Диакон: И во веки веков.
Хор: Аминь.
~8~

Трйсфятпе
Хор: Святый Боже, Святый Крепкий, Святый Безсмертный, помилуй нас. (Трижды)
Слава Отцу и Сыну и Святому Духу, и ныне и присно и во веки веков. Аминь.
Святый Безсмертный, помилуй нас.
Святый Боже, Святый Крепкий, Святый Безсмертный, помилуй
нас.
На Пасху, в праздники Рождества Христова, Богоявления, в Лазареву и Великую Субботы, во все дни Пасхальной Седмицы, в праздник Пятидесятницы вместо Трисвятого поется:
Елицы во Христа крестистеся, во Христа облекостеся, аллилуиа.
В праздник Воздвижения Креста Господня и в Неделю Крестопоклонную поется:
Кресту Твоему покланяемся, Владыко, и святое Воскресение Твое славим.
Диакон: Вонмем.
Иерей: Мир всем.
Чтец: И духови твоему.
Диакон: Премудрость.
Чтец: Прокимен, глас 3. Песнь Богородицы: Величит душа Моя Господа,/ и возрадовася дух Мой о Бозе Спасе Моем. (В Богородичные праздники)
Хор: Величит душа Моя Господа,/ и возрадовася дух Мой о Бозе Спасе Моем.
Чтец: Яко призре на смирение Рабы Своея, се бо отныне ублажат Мя вси роди.
Хор: Величит дем.
Чтец: Величит душа Моя Господа.
Хор: И возрадовася дух Мой о Бозе Спасе Моем.
~9~

Прокимны воскресные на Литургии Глас 1:
Буди, Господи, милость Твоя на нас,/ якоже уповахом на Тя. Глас 2:
Крепость моя и пение мое Господь,/ и бысть мне во спасение. Глас 3:
Пойте Богу нашему, пойте,/ пойте Цареви нашему, пойте. Глас 4:
Яко возвеличишася дела Твоя, Господи,/ вся премудростию
сотворил еси.
Ты, Господи, сохраниши ны/ и соблюдеши ны от рода сего и
во век.
Спаси, Господи, люди Твоя/ и благослови достояние Твое.
Глас 7:
Господь крепость людем Своим даст,/ Господь благословит
Глас 5:
Глас 6:
люди Своя миром.
Помолитеся и воздадите/ Господеви Богу нашему.
Прокимны дневные (будничные) В понедельник, глас 4:
Творяй Ангелы Своя духи,/ и слуги Своя пламень огненный. Во вторник, глас 7:
Возвеселится праведник о Господе/ и уповает на Него.
В среду, глас 3:
Величит душа Моя Господа/ и возрадовася дух Мой о Бозе
Глас 8:
Спасе Моем.
Во всю землю изыде вещание их,/ и в концы вселенныя
глаголы их.
В пятницу, глас 7;
Возносите Господа Бога нашего,/ и покланяйтеся подножию
ногу Его, яко свято есть.
В четверг, глас 8:
~ 10 ~

В субботу, глас 8:
Веселитеся о Господе/ и радуйтеся праведнии. Заупокойный, глас 6:
Души их/ во благих водворятся.
Диакон: Премудрость,
Чтец: Деяний святых апостол чтение (или: Соборнаго
Послания Петрова (или: Иоаннова, причем не принято говорить, какое это послание - первое, второе или третье) чтение; или: К римляном (К коринфяном; К галатом; К Тимофею и т. п.) послания святаго апостола Павла чтение).
Диакон: Вонмем.
Чтец читает Апостол.
Иерей: Мир ти.
Чтец: И духови твоему.
Диакон: Премудрость.
Чтец: Аллилуиа, аллилуиа, аллилуиа, глас 8 (или иной глас). Хор: Аллилуиа, аллилуиа, аллилуиа.
Чтец: Слыши, Дщи, и виждь, и приклони ухо Твое.
Хор: Аллилуиа, аллилуиа, аллилуиа.
Чтец: Лицу Твоему помолятся богатии людстии.
Хор: Аллилуиа, аллилуиа, аллилуиа.
Молитва перед чтением Евангелия:
Иерей: Возсияй в сердцах наших, Человеколюбче Владыко, Твоего богоразумия нетленный Свет, и мысленный наша отверзи очи, во евангельских Твоих проповеданий разумение; вложи в нас и страх блаженных Твоих заповедей, да, плотскйя похоти вся поправше, духовное жительство пройдем, вся, яже ко благоугождению Твоему, и мудрствующе, и деюще.Ты бо еси Просвещение душ и телес наших, Христе Боже, и Тебе славу возсылаем, со Безначальным Твоим Отцем, и Всесвятым и Благим и Животворящим Твоим Духом, ныне и присно и во веки веков. Аминь.
Диакон: Благослови, владыко, благовестителя святаго апостола и евангелиста (имярек евангелиста).
Иерей: Бог, молитвами святаго, славнаго, всехвальнаго апостола и евангелиста (имярек) да даст тебе глагол, благо-
~ 11 ~

вествующему силою многою, во исполнение Евангелия Воз- любленнаго Сына Своего, Господа нашего Иисуса Христа.
Подает диакону Святое Евангелие.
Диакон: Аминь.
Иерей: Премудрость, прости, услышим Святаго Евангелия. Мир всем.
Хор: И духови твоему.
Диакон: От (имярек евангелиста) Святаго Евангелия чтение. Хор: Слава Тебе, Господи, слава Тебе.
Иерей: Вонмем.
Диакон читает Святое Евангелие.
Иерей: Мир ти, благовествующему. Хор: Слава Тебе, Господи, слава Тебе.
Сузубая елтеойя
Диакон: Рцем вси от всея души и от всего помышления нашего рцем.
Хор: Господи, помилуй.
Диакон: Господи Вседержителю, Боже отец наших, молим Ти ся, услыши и помилуй.
Хор: Господи, помилуй.
Диакон: Помилуй нас, Боже, по велицей милости Твоей, молим Ти ся, услыши и помилуй.
Хор: Господи, помилуй. (Трижды на каждое прошение) Диакон: Еще молимся о Великом Господине и Отце нашем Святейшем Патриарсе Кирилле, и о Господине нашем Пре- освященнейшем (имя епархиального архиерея), и всей во Христе братии нашей.
~ 12 ~

Еще молимся о Богохранимей стране нашей, властех и воинстве ея, да тихое и безмолвное житие поживем во всяком благочестии и чистоте.
Еще молимся о братиях наших, священницех, священно- монасех и всем во Христе братстве нашем.
Еще молимся о блаженных и приснопамятных святейших патриарсех православных, и создателех святаго храма сего (если в монастыре: святыя обители сея,), и о всех прежде почивших отцех и братиях, зде лежащих и повсюду, православных.
Еще молимся о милости, жизни, мире, здравии, спасении, посещении, прощении и оставлении грехов рабов Божиих, братии святаго храма сего (если в монастыре: святыя обители сея). Еще молимся о плодоносящих и добродеющих во святем и всечестнѐм храме сем (если в монастыре: святой обите¬ли сей,), труждающихся, поющих и предстоящих людех, ожидающих от Тебе великия и богатыя милости.
Молитва прилежного моления
Иерей: Господи, Боже наш, прилежное сие моление приими от Твоих раб, и помилуй нас, по множеству милости Твоея, и щедроты Твоя низпосли на ны и на вся люди Твоя, чающия от Тебе богатыя милости.
Иерей: Яко милостив и Человеколюбец Бог еси, и Тебе славу возсылаем, Отцу и Сыну и Святому Духу, ныне и присно и во веки веков.
Хор: Аминь.
Заурплпйоая елтеойя
Диакон: Помилуй нас, Боже, по велицей милости Твоей, молим Ти ся, услыши и помилуй.
~ 13 ~

Хор: Господи, помилуй. (Трижды на каждое прошение) Диакон: Еще молимся о упокоении душ усопших рабов Божиих (имена) и о еже проститися им всякому прегрешению, вольному же и невольному.
Яко да Господь Бог учинит души их, идеже праведнии упокояются.
Милости Божия, Царства Небеснаго и оставления грехов их у Христа, Безсмертнаго Царя и Бога нашего, просим.
Хор: Подай, Господи. Диакон: Господу помолимся. Хор: Господи, помилуй.
Иерей: Боже духов и веяния плоти, смерть поправый и диавола упразднивый, и живот миру Твоему даровавый! Сам, Господи, покой души усопших раб Твоих (имена) в месте светле, в месте злачне, в месте покойне, отнюдуже отбеже болезнь, печаль и воздыхание; всякое согрешение, содеянное ими словом, или делом, или помышлением, яко Благий Человеколюбец Бог, прости. Яко несть человек, иже жив будет и не согрешит; Ты бо един токмо без греха, правда Твоя — правда во веки, и слово Твое — истина.
Иерей: Яко Ты еси Воскресение, и живот, и покой усопших раб Твоих (имена), Христе Боже наш, и Тебе славу возсылаем, со Безначальным Твоим Отцем и Пресвятым и Благим и Животворящим Твоим Духом, ныне и присно и вовеки веков. Хор: Аминь.
Царские врата закрываются.
Елтеойя пб пзмахеооци
Диакон: Помолитеся, оглашеннии, Господеви.
Хор: Господи, помилуй. (На каждое прошение) Диакон: Вернии, о оглашенных помолимся, да Господь помилует их.
Огласит их словом истины.
~ 14 ~

Открыет им Евангелие правды.
Соединит их Святей Своей, Соборней и Апостольстей Церкви. Спаси, помилуй, заступи и сохрани их, Боже, Твоею бла- годатию.
Оглашеннии, главы ваша Господеви приклоните.
Хор: Тебе, Господи.
Молитва об оглашенных
Иерей: Господи Боже ваш, Иже на высоких живый и на смиренный призираяй, Иже спасение роду человеческому низпославый — Единороднаго Сына Твоего и Бога, Господа нашего Иисуса Христа. Призри на рабы Твоя оглашенныя, подклоньшия Тебе своя выя, и сподоби я во время благополучное бани пакибытия, оставления грехов и одежди нетления, соедини их Святей Твоей Соборней и Апостольстей Церкви и сопричти их избранному Твоему стаду.
Иерей: Да и тии с нами славят пречестное и великолепое Имя Твое, Отца и Сына и Святаго Духа, ныне и присно и во веки веков.
Хор: Аминь.
мйтурзйя фероци
Диакон: Елицы оглашеннии, изыдите, оглашеннии, изыдите; елицы оглашеннии, изыдите. Да никто от оглашенных, елицы вернии, паки и паки миром Господу помолимся.
Хор: Господи, помилуй.
Диакон: Заступи, спаси, помилуй и сохрани нас, Боже, Твоею благодатию.
Хор: Господи, помилуй.
Диакон: Премудрость.
Молитва верных первая
Иерей: Благодарим Тя, Господи Боже сил, сподобившего нас предстати и ныне Святому Твоему Жертвеннику и припасти ко щедротам Твоим о наших гресех и о людских неведениих. Приими, Боже, моление наше, сотвори ны достойны быти, еже приносити Тебе моления и мольбы и Жертвы Безкровныя о всех людех Твоих; и удовли нас, ихже
~ 15 ~

положил еси в службу Твою сию, силою Духа Твоего Святаго, неосужденно и непреткновенно, в чистем свидетельстве совести нашея, призывати Тя на всякое время и место. Да, послушав нас, милостив нам будеши во множестве Твоея благости.
Иерей: Яко подобает Тебе всякая слава, честь и поклонение, Отцу и Сыну и Святому Духу, ныне и присно и во веки веков. Хор: Аминь.
Диакон: Паки и паки миром Господу помолимся.
Хор: Господи, помилуй. (На каждое прошение)
Диакон: О свышнем мире и спасении душ наших Господу помолимся.
О мире всего мира, благостоянии святых Божиих Церквей и соединении всех Господу помолимся.
О святем храме сем и с верою, благоговением и страхом Божиим входящих в онь Господу помолимся.
О избавитися нам от всякия скорби, гнева и нужды Господу помолимся.
Заступи, спаси, помилуй и сохрани нас, Боже, Твоею бла- годатию.
Диакон: Премудрость.
Молитва верных вторая
Иерей: Паки и многажды Тебе припадаем и Тебе молимся, Благий и Человеколюбче, яко да, призрев на моление наше, очистиши наша души и телеса от всякия скверны плоти и духа и даждь нам неповинное и неосужденное предстояние Святаго Твоего Жертвенника. Даруй же, Боже, и молящимся с нами преспеяние жития, и веры, и разума духовнаго. Даждь им, всегда со страхом и любовию служащим Тебе, неповинно и неосужденно причаститися Святых Твоих Тайн, и Небеснаго Твоего Царствия сподобитися.
Иерей: Яко да под державою Твоею всегда храними, Тебе славу возсылаем, Отцу и Сыну и Святому Духу, ныне и присно и во веки веков.
Хор: Аминь. (Протяжно)
Отверзаются Царские врата.
~ 16 ~

Херуфйнслая ресоь
Хор: Иже Херувимы тайно образующе и Животворящей
Троице Трисвятую песнь припевающе, всякое ныне житейское
отложим попечение.
Во время пения Херувимской песни диакон, взяв благословение от иерея, Читая про себя 50-й псалом, кадит Святой
Престол, Святой Жертвенник, алтарь, затем Царскими вратами выходит на солею и кадит иконостас: входит в алтарь, кадит иерея и предстоящих там; опять выходит на ам¬вон и кадит лики и молящихся. Иерей во время каждения читает тайно молитву Херувимской песни.
Иерей: Никтоже достоин от связавшихся плотскими похотьми и сластьми приходити, или приближитися, или служити Тебе, Царю славы; еже бо служити Тебе — велико и страшно и самем Небесным Силам. Но обаче, неизреченнаго ради и безмернаго Твоего человеколюбия, непреложно и неизменно был еси Человек и Архиерей нам был еси, и служебный сея и Безкровныя Жертвы священнодействие предал еси нам, яко Владыка всех. Ты бо един, Господи Боже наш, владычествуе-ши Небесными и земными, Иже на Престоле Херувимсте носимый, Иже Серафимов Господь и Царь Израилев, Иже един Свят и во святых почиваяй. Тя убо молю, единаго Блага- го и Благопослушливаго: призри на мя, грешнаго и непотребнаго раба Твоего, и очисти мою душу и сердце от совести лукавыя, и удовли мя, силою Святаго Твоего Духа, облеченна благодатию священства, предстати святей Твоей сей трапезе и священнодействовати Святое и Пречистое Твое Тело и Честную Кровь. К Тебе бо прихожду, приклонь мою выю, и молю Ти ся, да ие отвратиши лица Твоего от мене, ниже отринеши мене от отрок Твоих. Но сподоби принесѐнным Тебе быти много, грешным и недостойным рабом Твоим, Даром сим. Ты бо еси приносяй и приносимый, и приемляй и раздаваемый, Христе Боже наш, и Тебе славу возсылаем, со Безначальным Твоим Отцем и Пресвятым и Благим и Животворящим
Твоим Духом, ныне и присно и во веки веков. Аминь.
По окончании чтения иереем этой молитвы, а диаконом — каждения, оба они с умилением трижды вполголоса читают Херувимскую песнь (на Литургии св. Василия Великого в Великий Четверг читают: «Вечери Твоея Тайныя...», а в Великую Субботу - «Да молчит всякая плоть человеча...» ), затем отходят к Святому Жертвеннику.
Иерей кадит Святое Предложение, молясь тайно: Боже, очисти мя, грешнаго, (Трижды )
Диакон: Возьми, владыко.
Иерей, возлагая на его плечо воздух: Возьмите руки ваша во Святая и благословите Господа.
Иерей и диакон, взяв дискос и Чашу, исходят из алтаря северными вратами на солею.
~ 17 ~

Фемйлйй фипд
Диакон: Великаго Господина и Отца нашего Кирилла, Святейшего Патриарха Московского и всея Руси, и Господина нашего Преосвященнейшаго (имя епархиального архиерея), да помянет Господь Бог во Царствии Своем всегда, ныне и присно и во веки веков.
Иерей: Преосвященныя митрополиты, архиепископы и епископы, и весь священнический и монашеский чин, и причет церковный, братию святаго храма сего. Всех вас правосла- вных христиан да помянет Господь Бог во Царствии Своем всегда, ныне и присно и во веки веков.
Хор: Аминь. Яко да Царя всех подымем, ангельскими невидимо дориносима чинми. Аллилуиа, аллилуиа, аллилуиа.
Иерей и диакон входят, в алтарь.
Диакон: Да помянет Господь Бог священство твое во Царствии Своем.
Иерей: Да помянет Господь Бог священнодиаконство твое во Царствии Своем, всегда, ныне и присно и во веки веков.
Иерей, поставляя Дары на Престол, читает тропари: «Благообразный Иосиф...», «Во гробе плотски, во аде же с душею, яко Бог...», «Яко Живоносец, яко Рая краснейший...» Ублажи, Господи, благоволением Твоим Сиона, и да созиждутся стены Иерусалимския; тогда благоволиши жертву правды, возношение н всесожегаемая, тогда возложат на олтарь Твой тельцы.
Иерей диакону: Помолися о мне, брате и сослужителю.
Диакон: Дух Святый найдет на тя, н сила Вышняго осенит тя.
Иерей: Тойже Дух содействует нам вся дни живота нашего.
Диакон: Помяни и мя, владыко святый.
Иерей, благословляя: Да помянет тя Господь Бог во Царствии Своем всегда, ныне и присно и во веки веков.
Диакон: Аминь.
Закрываются Царские врата и завеса.
~ 18 ~

Прпсйтемьоая елтеойя
Диакон: Исполним молитву нашу Господеви.
Хор: Господи, помилуй. (На каждое прошение)
Диакон: О предложенных честных Дарех Господу помолимся. О святем храме сем и с верою, благоговением и страхом Божиим входящих в онь Господу помолимся.
О избавитися нам от всякия скорби, гнева и нужды Господу помолимся.
Заступи, спаси, помилуй и сохрани нас, Боже, Твоею бла- годатию.
Дне всего совершенна, свята, мирна и безгрешна у Господа просим.
Хор: Подай, Господи. (На каждое прошение)
Ангела мирна, верна наставника, хранителя душ и телес наших у Господа просим.
Прощения и оставления грехов и прегрешений наших у Гос- пода просим.
Добрых и полезных душам нашим и мира мирови у Господа просим.
Прочее время живота нашего в мире и покаянии скончати у Господа просим.
Христианския кончины живота нашего, безболезнены, непостыдны, мирны, и добраго ответа на Страшнем Судищи Христове просим.
Пресвятую, Пречистую, Преблагословенную, Славную Влады- чицу нашу Богородицу и Приснодеву Марию, со всеми святы- ми помянувше, сами себе, и друг друга, и весь живот наш Христу Богу предадим.
Хор: Тебе, Господи.
~ 19 ~

Иерей: Господи Боже Вседержителю, Едине Святе, приемляй жертву хваления от призывающих Тя всем сердцем! Приими и нас, грешных, моление, и принеси ко Святому Твоему Жертвеннику, и удовли нас приносити Тебе дары же и жертвы духовныя о наших гресех и о людских неведениих; и сподоби нас обрести благодать пред Тобою, еже быти Тебе благоприятней жертве нашей и вселитися Духу благодати Твоея Благому в нас, и на предлежащих Дарех сих, и на всех людех Твоих.
Иерей: Щедротами Единороднаго Сына Твоего, с Нимже благословен еси, со Пресвятым и Благим и Животворящим Твоим Духом, ныне и присно и во веки веков.
Хор: Аминь.
Иерей: Мир всем.
Хор: И духови твоему.
Диакон: Возлюбим друг друга, да единомыслием исповемы. Хор: Отца и Сына и Святаго Духа, Троицу Единосущную и
Нераздельную.
Иерей, поклоняясь: Возлюблю Тя, Господи, крепосте моя, Господь утверждение мое и прибежище мое. (Трижды)
Диакон: Двери, двери, премудростию вонмем. Открывается завеса Царских врат.
Сйнфпм ферц
Хор и народ: Верую во Единаго Бога Отца Вседержителя, Творца небу и земли, видимым же всем и невидимым. И во Единаго Господа Иисуса Христа, Сына Божия, Единородного, Иже от Отца рожденнаго прежде всех век. Света от Света, Бога истинна от Бога истинна, рожденна, несотворенна, Единосущна Отцу, Имже вся быша. Нас ради, человек, и нашего ради спасения сшедшаго с Небес, и воплотившагося от Духа Свята и Марии Девы, и вочеловечшася. Распятаго же за ны при Понтийстем Пилате, и страдавша, и погребенна. И
~ 20 ~

воскресшаго в третий день по Писанием. И восшедшаго на Небеса, и седяща одесную Отца. И паки грядущаго со славою судйти живым и мертвым, Егоже Царствию не будет конца. И в Духа Святаго, Господа Животворящаго, Иже от Отца исходящаго, Иже со Отцем и Сыном спокланяема и сславима, глаголавшаго пророки. Во едину Святую Соборную и Апостольскую Церковь. Исповедую едино Крещение во остав- ление грехов. Чаю воскресения мертвых и жизни будущаго века. Аминь.
Диакон: Станем добре, станем со страхом, вонмем, Святое Возношение в мире приносити.
Хор: Милость мира, жертву хваления .
Иерей: Благодать Господа нашего Иисуса Христа, и любы
Бога и Отца, и причастие Святаго Духа буди со всеми вами. Хор: И со духом твоим.
Иерей: Горе имеем сердца.
Хор: Имамы ко Господу.
Иерей: Благодарим Господа.
Хор: Достойно и праведно есть покланятися Отцу и Сыну и
Святому Духу, Троице Единосущней и Нераздельней.
Иерей: Достойно и праведно Тя пети, Тя благословити, Тя хвалити, Тя благодарити, Тебе покланятися на всяком месте владычествия Твоего. Ты бо еси Бог Неизречен, Недоведом, Невидим, Непостижим, присно Сый, такожде Сый, Ты, и Единородный Твой
~ 21 ~

Сын, и Дух Твой Святый. Ты от небытия в бытие нас привел еси, и отпадшия возставил еси паки, и не отступил еси, вся творя, дондеже нас на Небо возвел еси и Царство Твое даровал еси будущее. О сих всех благодарим Тя, и Единороднаго Твоего Сына, и Духа Твоего Святаго о всех, ихже вемы и ихже не вемы, явленных и неявленных благодеяниих, бывших на нас. Благодарим Тя и о службе сей, юже от рук наших прияти изволил еси, аще и предстоят Тебе тысящи Архангелов и тьмы Ангелов, Херувими и Серафими, шестокрилатии, многоочитии, возвышающиеся, пернатии.
Иерей: Победную песнь поюще, вопиюще, взывающе и глаголюще.
Хор: Свят, Свят, Свят Господь Саваоф, исполнь Небо и земля славы Твоея; осанна в вышних, благословен Грядый во Имя Господне, осанна в вышних.
Иерей: С сими и мы блаженными Силами, Владыко Человеколюбче, вопием и глаголем: Свят еси и Пресвят, Ты, и Единородный Твой Сын, и Дух Твой Святый; Свят еси и Пресвят, и великолепна слава Твоя; Иже мир Твой тако возлюбил еси, якоже Сына Своего Единороднаго дати, да всяк веруяй в Него не погибнет, но имать живот вечный. Иже, пришед и все еже о нас смотрение исполнив, в нощь, в нюже предаяшеся, паче же Сам Себе предаяшее за мирский живот, прием хлеб во Святыя Своя и Пречистыя и непорочныя руки, благодарив и благословив, освятив, преломив, даде святым Своим учеником и апостолом, рек:
Иерей: Приимите, ядите, Сие есть Тело Мое, еже за вы ломимое во оставление грехов.
Хор: Аминь.
Иерей: Подобие и Чашу по вечери, глаголя:
Иерей: Пийте от нея вси, Сия есть Кровь Моя Новаго Завета, яже за вы и за многи изливаемая во оставление грехов.
Хор: Аминь.
Иерей: Поминающе убо спасительную сию заповедь и вся, яже о нас бышая: Крест, Гроб, тридневное Воскресение, на Небеса восхождение, одесную седение, второе и славное паки пришествие.
~ 22 ~

Иерей: Твоя от Твоих Тебе приносяще о всех и за вся. Хор: Тебе поем, Тебе благословим, Тебе благодарим,
Господи, и молим Ти ся, Боже Наш. (Протяжно)
Иерей: Еще приносим Ти словесную сию и безкровную службу, и просим, и молим, и мили ся деем, низпосли Духа Твоего Святаго на ны и на предлежащия Дары сия. Господи, Иже Пресвятаго Твоего Духа в третий час Апостолом Твоим низпославый, Того, Благий, не отыми от нас, но обнови нас, молящих Ти ся. (Трижды, поклоняясь перед престолом)
* На Литургии св. Василия Великого эти возгласы предваряются словами:
Дадѐ святым Своим учеником и апостолом, рек.
Диакон (после первого произнесения тропаря): Сердце чисто созижди во мне, Боже, и дух прав обнови во утробе моей.
Диакон (после второго произнесения тропаря): Не отвержи мене от лица Твоего и Духа Твоего Святаго не отымй от мене.
Диакон (указывая орарем на Святой Хлеб): Благослови, владыко, Святый Хлеб.
Иерей (благословляя): И сотвори убо Хлеб Сей Честное Тело Христа Твоего.
Диакон: Аминь. Благослови, владыко, Святую Чашу.
Иерей (благословляя): А еже в Чаши сей, Честную Кровь Христа Твоего.
Диакон: Аминь. Благослови, владыко, обоя (указывая орарем на Святые Дары).
Иерей (благословляя Святые Дары вместе): Преложив Духом Твоим Святым.
Диакон: Аминь, аминь, аминь.
Диакон: Помяни мя, святый владыко, грешнаго.
Иерей: Да помянет тя Господь Бог во Царствии Своем всегда, ныне и присно и во веки веков.
Диакон: Аминь.
Иерей: Якоже быти причащающимся, во трезвение души, во оставление грехов, в приобщение Святаго Твоего Духа, во исполнение Царствия Небеснаго, в дерзновение еже к Тебе, не в суд или во осуждение. Еще приносим Ти словесную сию службу о иже в вере почивших праотцех, отцех, патриарсех, пророцех, апостолех, проповедницех, евангелистех, мученицех, исповедницех, воздержницех и о всяком дусе праведней, в вере скончавшемся.
Иерей: Изрядно о Пресвятей, Пречистей, Преблагоеловен- ней, Славней Владычице нашей Богородице и Приснодеве Марии.
~ 23 ~

Хор: Достойно есть, яко воистину блажити Тя, Богородицу, Присноблаженную и Пренепорочную и Матерь Бога нашего. Честнейшую Херувим и Славнейшую без сравнения Серафим, без изстления Бога Слова рождшую, сущую Богородицу Тя величаем.
На Литургии св. Василия Великого вместо «Достойно есть...» поется:
О Тебе радуется, Благодатная, всякая тварь, ангельский собор и человеческий род, освященный Храме и Раю словесный, девственная похвало, из Неяже Бог воплотйся и младенец бысть, прежде век сый Бог наш; ложесна бо Твоя престол сотвори и чрево Твое пространнее небес содела. О Тебе радуется, Благодатная, всякая тварь, слава Тебе.
Иерей: О святем Иоанне Пророце, Предтечи и Крестители, о святых славных и всехвальных Апостолех, о святем (имя), егоже и память совершаем, и о всех святых Твоих, ихже молитвами посети нас, Боже. И помяни всех усопших о надежди воскресения жизни вечныя (имена).И упокой их, идеже присещает свет лица Твоего, Еще молим Тя, помяни, Господи, всякое епископство православных, право правящих слово Твоея истины, всякое пресвитерство, во Христе диаконство и всякий священнический чин. Еще приносим Ти словесную сию службу о вселенней, о Святей, Соборней и Апостольстей Церкви, о иже в чистоте и честнем жительстве пребывающих; о Богохранимей стране нашей, властех и воинстве ея. Даждь им, Господи, мирное правление, да и мы в тишине их тихое и безмолвное житие поживем, во всяком благочестии и чистоте.
Иерей: В первых помяни, Господи, Великаго Господина и Отца нашего Кирилла, Святейшаго Патриарха Московскаго и всея Руси, и Господина нашего Преосвященнейшаго (имя епархиального архиерея), ихже даруй святым Твоим Церквам в мире, целых, честных, здравых, долгоденствующих, право правящих слово Твоея истины.
Хор: И всех и вся.
Иерей: Помяни, Господи, град сей, в немже живем (или: весь сию, в нейже живем, или:
обитель сию), и всякий град и страну и верою живущих в них. Помяни, Господи, ~ 24 ~

плавающих, путешествующих, недугующих, страждущих, плененных, и спасение их. Помяни, Господи, плодоносящих и добротворящих во святых Твоих церквах, и поминающих убогия, и на вся ны милости Твоя ни зпосли (имена живых).
Иерей: И даждь нам едиными усты и единым сердцем сла- вити и воспевати Пречестное и Великолепое Имя Твое, Отца и Сына и Святаго Духа, ныне и присно и во веки веков.
Хор: Аминь.
Иерей: И да будут милости Великаго Бога и Спаса нашего
Иисуса Христа со всеми вами. Хор: И со духом твоим.
Прпсйтемьоая елтеойя
Диакон: Вся святыя помянувше, паки и паки миром Господу помолимся.
Хор: Господи, помилуй. (На каждое прошение)
Диакон: О принесенных и освященных Честных Дарех Господу помолимся.
Яко да Человеколюбец Бог наш, приемь я во святый, и пренебесный, и мысленный Свой Жертвенник, в воню благоухания духовнаго, возниспослет нам Божественную благодать и дар Святаго Духа, помолимся.
О избавитися нам от всякия скорби, гнева и нужды Гос¬поду помолимся.
Заступи, спаси, помилуй и сохрани нас, Боже, Твоею бла- годатию.
Дне всего совершенна, свята, мирна и безгрешна у Господа просим.
~ 25 ~

Хор: Подай, Господи. (На каждое прошение)
Ангела мирна, верна наставника, хранителя душ и телес наших у Господа просим.
Прощения и оставления грехов и прегрешений наших у Господа просим.
Добрых и полезных душам нашим и мира мирови у Господа просим.
Прочее время живота нашего в мире и покаянии скончати у Господа просим.
Христианския кончины живота нашего, безболезнены, не- постыдны, мирны, и добраго ответа на Страшнем Судищи Христове просим.
Соединение веры и причастие Святаго Духа испросивше, сами себе, и друг друга, и весь живот наш Христу Богу предадим.
Хор: Тебе, Господи.
Иерей: Тебе предлагаем живот наш весь и надежду, Владыко Человеколюбче, и просим, и молим, и мили ся деем: сподоби нас причаститися Небесных Твоих и Страшных Тайн, сея священныя и духовный Трапезы, с чистою совестию, во оставление грехов, в прощение согрешений, во общение Духа Святаго, в наследие Царствия Небеснаго, в дерзновение еже к Тебе, не в суд или во осуждение.
Иерей: И сподоби нас, Владыко, со дерзновением, нео- сужденно смети призывати Тебе, Небеснаго Бога Отца, и глаголати.
Хор: Отче наш, Иже еси на Небесех! Да святится Имя Твое, да прийдет Царствие Твое, да будет воля Твоя, яко на небеси и на земли. Хлеб наш насущный даждь нам днесь, и остави нам долги наша, якоже и мы оставляем должником нашим; и не введи нас во искушение, но избави нас от лукаваго.
~ 26 ~

Иерей: Яко Твое есть Царство, и сила, и слава, Отца и Сына и Святаго Духа, ныне и присно и во веки веков.
Хор: Аминь.
Иерей: Мир всем.
Хор: И духови твоему.
Диакон: Главы ваша Господеви приклоните.
Хор: Тебе, Господи. (Протяжно)
Иерей: Благодарим Тя, Царю невидимый, Иже неисчетною Твоею силою вся содетельствовал еси и множеством милости Твоея от небытия в бытие вся привел еси. Сам, Владыко, с Небесе призри на подклоньшия Тебе главы своя; не бо подклониша плоти и крови, но Тебе, Страшному Богу. Ты убо, Владыко, предлежащая всем нам во благое изравняй, по коегождо своей потребе: плавающим сплавай, путешествующим спутешествуй, недугующия исцели, Врачу душ и телес.
Иерей: Благодатию, и щедротами, и человеколюбием Едино- роднаго Сына Твоего, с Нимже благословен еси, со Пресвя- тым и Благим и Животворящим Твоим Духом, ныне и присно и во веки веков.
Хор: Аминь. (Протяжно)
Иерей: Вонми, Господи Иисусе Христе, Боже наш, от святаго жилища Твоего и от Престола славы Царствия Твоего, и прииди во еже освятити нас, Иже горе со Отцем седяй и зде нам невидимо спребываяй, и сподоби державною Твоею рукою преподати нам Пречистое Тело Твое и Честную Кровь, и нами — всем людем.
Диакон: Вонмем.
~ 27 ~

Закрывается завеса Царских врат.
Иерей: Святая святым.
Хор: Един Свят, един Господь Иисус Христос, во славу Бога Отца. Аминь.
Хвалите Господа с небес, хвалите Его в вышних. Аллилуиа, аллилуиа, аллилуиа. (Причастен в воскресенье)
Прйчастоц
В понедельник:
Творяй Ангелы Своя духи, и слуги Своя пламень огненный.
Во вторник:
В память вечную будет праведник, от слуха зла не убоится.
В среду:
Чашу спасения прииму и Имя Господне призову.
В четверг:
Во всю землю изыде вещание их, и в концы вселенныя
глаголы их.
В пятницу:
Спасение соделал еси посреде земли, Боже.
В субботу:
Радуйтеся, праведнии, о Господе, правым подобает похвала.
Заупокойный:
Блажени, яже избрал и приял еси, Господи, и память их в род
и род.
В праздники Богородичные:
Чашу спасения прииму и Имя Господне призову.
В дни памяти святых:
В память вечную будет праведник, от слуха зла не убоится.
~ 28 ~

В праздники Апостолов:
Во всю землю изыде вещание их, и I в концы вселенныя глаголы их.
В это время происходит Причащение священнослужителей в алтаре. После этого отверзаются Царские врата. Диакон, выйдя вместе с иереем на солею через Царские врата и подняв Святую Чашу вверх, произносит:
Диакон: Со страхом Божиим и верою приступите!
Хор: Благословен Грядый во Имя Господне, Бог Господь и
явися нам.
Иерей (и с ним все, желающие причаститься): Верую, Господи, и исповедую, яко Ты еси воистину Христос, Сын Бога Живаго, пришедый в мир грешныя спасти, от нихже первый есмь аз. Еще верую, яко Сие самое есть Пречистое Тело Твое, и Сия самая есть Честная Кровь Твоя. Молюся убо Тебе: помилуй мя, и прости ми прегрешения моя, вольная и неволь- ная, яже словом, яже делом, яже ведением и неведением, и сподоби мя неосужденно причаститися Пречистых Твоих Таинств, во оставление грехов и в жизнь вечную. Аминь. Вечери Твоея Тайныя днесь, Сыне Божий, причастника мя приими; не бо врагом Твоим тайну повем, ни лобзания ти дам яко Иуда, но яко разбойник исповедаю Тя: помяни мя, Господи, во Царствии Твоем.
Да не в суд или во осуждение будет мне причащение Святых Твоих Тайн, Господи, но во исцеление души и тела.
Причащая мирян, иерей глаголет: Причащается раб Божий (имя) Честнаго и Святаго Тела и Крове Господа и Бога и Спаса
~ 29 ~

нашего Иисуса Христа, во оставление грехов своих и в жизнь вечную.
Хор (во время причащения): Тело Христово приимите, Источника безсмертнаго вкусите.
Хор (по окончании причащения): Аллилуиа, аллилуиа, аллилуиа.
* В пасхальную Седмицу поется «Христос воскресе...»
** При причащении младенцев иерей произносит: «Честныя и Святы я Крове Господа и Бога и Спаса нашего Иисуса Христа причащается младенец (имя) в жизнь вечную».
*** В Великий Четверг поется «Вечери Твоея тайныя...», в пасхальную Седмицу поется «Христос воскресе... »
После причащения мирян иерей с диаконом входят, в алтарь. Иерей ставит Святую Чашу на Святой Престол, а диакон, взяв Святой Дискос, читает воскресные (пасхальные) песнопения: «Воскресение Христово видевше..,», «Светися, светися...», «О Пасха велия и священнейшая, Христе..»
Затем диакон опускает в Потир частицы, вынутые из просфор, лежащих на Дискосе, со словами: Отмый, Господи, грехи поминавшихся зде Кровию Твоею Честною, молитвами святых Твоих.
Иерей (благословляя народ): Спаси, Боже, люди Твоя и
благослови достояние Твое.
Во время пения «Видехом свет истинный...» иерей кадит трижды Святые Дары, говоря: Иерей: Вознесйся на Небеса, Боже, и по всей земли слава Твоя.
Хор: Видехом Свет Истинный, прияхом Духа Небеснаго, обре- тохом веру истинную, Нераздельней Троице покланяемся, Та бо нас спасла есть.
Иерей: Благословен Бог наш.
Иерей (обратившись лицом к народу со Святой Чашей): Всегда, ныне и присно и во веки веков.
~ 30 ~

Хор: Аминь. Да исполнятся уста наша хваления Твоего, Господи, яко да поем славу Твою, яко сподобил еси нас причаститися Святым Твоим, Божественным, Безсмертным и Животоворящим Тайнам; соблюди нас во Твоей святыни, весь день поучатися правде Твоей. Аллилуиа, аллилуиа, аллилуиа .
Диакон: Прости приимше Божественных, Святых, Пречистых, Безсмертных, Небесных и Животворящих, Страшных Христовых Тайн, достойно благодарим Господа.
Хор: Господи, помилуй. (На каждое прошение)
Диакон: Заступи, спаси, помилуй и сохрани нас, Боже, Твоею благодатию .
День весь совершен, свят, мирен и безгрешен испросивше, сами себе, и друг друга, и весь живот наш Христу Богу предадим.
Хор: Тебе, Господи.
Иерей, сложив Антиминс и держа вертикально Святое Евангелие, делает им знак креста
над Антиминсом и произносит:
Иерей: Яко Ты еси Освящение наше, и Тебе славу возсы- лаем, Отцу и Сыну и Святому Духу, ныне и присно и во веки веков.
Хор: Аминь.
Иерей: С миром изыдем. Хор: О имени Господни.
~ 31 ~

Диакон: Господу помолимся. Хор: Господи, помилуй.
Мпмйтфа чаанфпооая
Иерей: Благословляяй благословящия Тя, Господи, и ос- вящаяй на Тя уповающия, спаси люди Твоя и благослови достояние Твое, исполнение Церкве Твоея сохрани, освяти любящия благолепие дому Твоего; Ты тех возпрослави Божественною Твоею силою, и не остави нас, уповающих на Тя. Мир мирови Твоему даруй, Церквам Твоим, священником, воинству и всем людем Твоим. Яко всякое даяние благо, и всяк дар совершен свыше есть, сходяй от Тебе, Отца Светов; и Тебе славу, и благодарение, и поклонение возсылаем, Отцу и Сыну и Святому Духу, ныне и присно и во веки веков.
Хор: Аминь. Буди Имя Господне благословено отныне и до века*. (Тржды)
Псампн 33**
Хор: Благословлю Господа на всякое время,/ выну хвала Его во устех моих./ О Господе похвалится душа моя,/ да услышат кротции, и возвеселятся./ Возвеличите Господа со мною,/ и вознесѐм Имя Его вкупе./ Взысках Господа, и услыша мя,/ и от всех скорбей моих избави мя. /Приступите к Нему и просветитеся,/ и лица ваша не постыдятся./ Сей нищий воззва, и Господь услыша и,/ и от всех скорбей его спасе и./ Ополчится ангел Господень окрест боящихся Его,/ и избавит
~ 32 ~

их. /Вкусите, и видите, яко благ Господь:/ блажен муж, иже уповает Нань. /Бойтеся Господа вси святии Его,/ яко несть лишения боящимся Его. /Богатии обнищаша и взалкаша:/ взыскающии же Господа //не лишатся всякаго блага.
* Па пасхальной Седмице поется «Христос воскресе... »
**На пасхальной Седмице вместо 33-го псалма поется многократно «Христос воскресе»
Иерей: Благословение Господне на вас, Того благодатию и человеколюбием, всегда, ныне и присно и во веки веков.
Хор: Аминь.
Иерей: Слава Тебе, Христе Боже, Упование наше, слава Тебе.
Хор: Слава Отцу и Сыну и Святому Духу, и ныне и присно и во веки веков. Аминь. Господи, помилуй. (Трижды) Благослови.
Иерей: Воскресый из мертвых, Христос, истинный Бог наш, молитвами Пречистыя Своея Матере, святых славных и всехвальных Апостол, иже во святых отца нашего Иоанна, архиепископа Константина града, Златоустаго и святаго (храма и святого, которого память в этот день), святых и праведных Богоотец Иоакима и Анны и всех святых, помилует и спасет нас, яко Благ и Человеколюбец.
Хор: Великаго Господина и Отца нашего Кирилла, Святейшаго Патриарха Московскаго и всея Руси, и Господина нашего Преосвященнейшаго (имя епархиального архиерея), братию святаго храма сего и вся православный христианы, Господи, сохрани их на многая лета.
~ 33 ~

КОНЕВ
И БОГУ НАХЕМУ СЛАФА.
~ 34 ~

