\pagestyle{empty}
\vspace*{1cm}
\begin{center}
\fontspec{PTS75F.ttf}
\fontspec{PTS75F.ttf}
\fontsize{24}{32}
\selectfont
Последование \\Божественной \\Литургии\\
\fontsize{18}{24}\selectfont
для общенародного пения
\end{center}
\vfill
\cleardoublepage
\setcounter{page}{1}
\sloppy
\Section{Литургия оглашенных}
% \diak{Благослови, владыко.}

\ierey{Благословенно Царство Отца и Сына и Святаго
Духа ныне и присно и во веки веков.}
\pagestyle{plain}

\xor{Аминь.}

\Subsection{Мирная ектения}

\diak{Миром Господу помолимся.}

\xor{Господи помилуй. \comment{(На каждое прошение.)}}

\diak{О свышнем мире и спасении душ наших Господу помолимся.}

\diakX{О мире всего мира, благостоянии Святых Божиих Церквей и соединении всех Господу помолимся.}

\diakX{О святем храме сем и с верою, благоговением и страхом Божиим входящих вонь Господу помолимся.}

\diakX{О Великом Господине и Отце нашем Святейшем Патриарсе
Кирилле,
и о Господине нашем Высокопреосвященнейшем митрополите Клименте,
честнем пресвитерстве, во Христе
диаконстве, о всем причте и людех Господу помолимся.}

\diakX{О Богохранимей стране нашей, властех и воинстве ея Господу помолимся.}

\diakX{О граде сем
%(или: о веси сей; если в монастыре, то: о святой обители сей)
, всяком граде, стране и верою живущих в них Господу помолимся.}

\diakX{О благорастворении воздухов, о изобилии плодов земных и временех мирных Господу помолимся.}

\diakX{О плавающих, путешествующих, недугующих,
страждущих, плененных и о спасении их Господу помолимся.}

\diakX{О избавитися нам от всякия скорби, гнева и нужды Господу помолимся.}

\diakX{Заступи, спаси, помилуй и сохрани нас, Боже, Твоею благодатию.}

\diakX{Пресвятую, Пречистую, Преблагословенную, Славную Владычицу
нашу Богородицу и Приснодеву Марию, со всеми святыми
помянувше, сами себе и друг друга, и весь живот наш Христу
Богу предадим.}

\xor{Тебе, Господи.}

% Тайная молитва первого антифона
%
% Иерей: Господи Боже наш, Егоже держава несказанна и слава
% непостижима, Егоже милость безмерна и человеколюбие
% неизреченно! Сам, Владыко, по благоутробию Твоему, призри на
% ны и на святый храм сей и сотвори с нами и молящимися с
% нами богатыя милости Твоя и щедроты Твоя.

\ierey{Яко подобает Тебе всякая слава, честь и поклонение,
Отцу и Сыну и Святому Духу, ныне и присно и во веки веков.}

\xor{Аминь.}

\Subsection{Первый антифон}

\xor{
Благослови, душе моя, Господа.\,/ Благословен еси, Господи.}

\xorX{Благослови, душе моя, Господа,\,/ и вся внутренняя моя Имя Святое Его.}

\xorX{Благослови, душе моя, Господа,\,/ и не забывай всех воздаяний Его.}

\xorX{Очищающаго вся беззакония твоя,\,/ исцеляющаго вся недуги твоя.}

\xorX{Избавляющаго от истления живот твой,\,/ венчающаго тя милостию и щедротами.}

\xorX{Исполняющего во благих желание твое,\,/ обновится, яко орля, юность твоя.}

\xorX{Творяй милостыни Господь,\,/ и судьбу всем обидимым.}

\xorX{Щедр и милостив Господь,\,/ долготерпелив и многомилостив.}

\xorX{Благослови, душе моя, Господа,\,/ и вся внутреняя моя,\,/ Имя Святое Его.\,// Благословен еси, Господи.}

\Subsection{Малая ектения}

\diak{Паки и паки миром Господу помолимся.}

\xor{Господи помилуй. \comment{(На оба прошения.)}}

\diak{Заступи, спаси, помилуй и сохрани нас, Боже, Твоею благодатию.}

\diakX{Пресвятую, Пречистую, Преблагословенную, Славную Владычицу
нашу Богородицу и Приснодеву Марию, со всеми святыми
помянувше, сами себе и друг друга, и весь живот наш Христу
Богу предадим.}

\xor{Тебе, Господи.}

% Тайная молитва второго антифона
%
% Иерей: Господи Боже наш! Спаси люди Твоя и благослови
% достояние Твое, исполнение Церкве Твоея сохрани, освяти
% любящия благолепие дому Твоего; Ты тех воспрослави
% Божественною Твоею силою и не остави нас, уповающих на Тя,

\ierey{Яко Твоя держава, и Твое есть Царство, и сила, и
слава, Отца и Сына и Святаго Духа, ныне и присно и во веки
веков.}

\xor{Аминь.}

\Subsection{Второй антифон}
\xor{Хвали, душе моя, Господа.}

\xorX{Восхвалю Господа в животе моем,\,/ пою Богу моему, дондеже есмь,}

\xorX{Не надейтеся на князи, на сыны человеческия,\,/ в нихже несть спасения.}

\xorX{Изыдет дух его, и возвратится в землю свою:\,/ в той день погибнут вся помышления его.}

\xorX{Блажен, емуже Бог Иаковль помощник его,\,/ упование его на Господа Бога своего.}

\xorX{Сотворшаго небо и землю,\,/ море и вся, яже в них.}

\xorX{Хранящаго истину в век,\,/ творящаго суд обидимым,\,/ дающаго пищу алчущим.}

\xorX{Господь решит окованныя,\,/ Господь умудряет слепцы.}

\xorX{Господь возводит низверженныя,\,/ Господь любит праведники.}

\xorX{Господь хранит пришельцы,\,/ сира и вдову приимет,\,/ и путь грешных погубит.}

\xorX{Воцарится Господь во век,\,// Бог твой, Сионе, в род и род.}

\xorX{Слава Отцу и Сыну и Святому Духу и ныне и присно и во веки веков. Аминь.}
\clearpage

\Subsection{Единородный Сыне}
\xor{Единородный Сыне и Слове Божий, Безсмертен Сый,\,/ и
изволивый спасения нашего ради\,/ воплотитися от Святыя
Богородицы и Приснодевы Марии,\,/ непреложно
вочеловечивыйся;\,/ распныйся же, Христе Боже, смертию
смерть Поправый,\,/ един Сый Святыя Троицы,\,//
спрославляемый Отцу и Святому Духу, спаси нас.  }

\Subsection{Малая ектения}

\diak{Паки и паки миром Господу помолимся.}

\xor{Господи помилуй. \comment{(На оба прошения, без паузы.)}}

\diak{Заступи, спаси, помилуй и сохрани нас, Боже, Твоею благодатию.}

\diakX{Пресвятую, Пречистую, Преблагословенную, Славную Владычицу
нашу Богородицу и Приснодеву Марию, со всеми святыми
помянувше, сами себе и друг друга, и весь живот наш Христу
Богу предадим.}

\xor{Тебе, Господи.}

%  Тайная молитва третьего антифона
%
% Иерей: Иже общия сия и согласный даровавый нам молитвы,
% Иже и двема или трем, согласующимся о имени Твоем, прошения
% подати обещавый! Сам и ныне раб Твоих прошения к полезному
% исполни, подая нам и в настоящем веце познание Твоея истины
% и в будущем живот вечный даруя.

\ierey{Яко Благ и Человеколюбец Бог еси, и Тебе славу возсылаем, Отцу и Сыну и Святому Духу, ныне и присно и во веки веков.}

\xor{Аминь.}

\clearpage
\Subsection{Третий антифон}
\xor{Во Царствии Твоем помяни нас, Господи,\,/ егда приидеши во
Царствии Твоем.}

\xorX{Блажени нищии духом,\,/ яко тех есть Царство Небесное.}

\xorX{Блажени плачущии,\,/ яко тии утешатся.}

\xorX{Блажени кротции,\,/ яко тии наследят землю.}

\xorX{Блажени алчущии и жаждущии правды,\,/ яко тии насытятся.}

\xorX{Блажени милостивии,\,/ яко тии помиловани будут.}

\xorX{Блажени чистии сердцем,\,/ яко тии Бога узрят.}

\xorX{Блажени миротворцы,\,/ яко тии сынове Божии нарекутся.}

\xorX{Блажени изгнани правды ради,\,/ яко тех есть Царство
Небесное.}

\xorX{Блажени есте, егда поносят вам,\,/ и изженут, и рекут всяк зол глагол на вы, лжуще Мене ради.}

\xorX{Радуйтеся и веселитеся,\,/ яко мзда ваша многа на Небесех.}

% Слава Отцу и Сыну /и Святому Духу.
% И ныне и присно /и во веки веков. Аминь.

% Молитва входа
% Иерей: Владыко Господи, Боже наш, уставивый на Небесех чины и воинства Ангел и Архангел в
% служение Твоея славы! Сотвори со входом нашим входу святых Ангелов быти, сослужащих нам и сословословящих Твою благость, Яко подобает Тебе всякая слава, честь и поклонение, Отцу и Сыну и Святому Духу, ныне и присно и во веки веков. Аминь.
%   Диакон: Благослови, владыко, святый вход.
%  Иерей (благословляя): Благословен вход святых Твоих, всегда, ныне и присно и во веки веков.

\diak{Премудрость, прости.}

% В дни Великих Господских праздников (Пасха, Рождество Христово, Богоявление, Сретение Господне, Вход
% Господень в Иерусалим, Вознесение Господне, Пятидесятница и День Святого Духа, Преображение,
%   Воздвижение Креста Господня) диакон возглашает также входной стих. В этом случае Входное не поется', а сразу тропарь и кондак праздника.
\xor{
Приидите, поклонимся и припадем ко Христу. Спаси ны, Сыне Божий, \comment{воскресый из мертвых}%
% \footnote{Вместо слов «воскресый из мертвых» поется:\\
% В будние дни — «во святых Дивен сый».\\
% В Богородичные двунадесятые праздники и в дни их попразднства — Молитвами Богородицы.\\
% В попразднство Рождества Христова — Рождейся от Девы.\\
% Богоявления — Во Иордане крестивыйся.\\
% Вознесения Господня — Вознесыйся во славе.\\
% Преображения — Преобразивыйся на горе.\\
% Воздвижения Креста Господня — Плотию распныйся.\\
% В попразднство Пятидесятницы вместо слов «Спаси ны, Сыне Божий, воскресый из мертвых» поется «Спаси ны, Утешителю Благий».%
% }
, поющия Ти: аллилуиа.}


\Subsection{Тропари и кондаки}
\Xcomment{В зависимости от дня поются тропари и кондаки.}

\begin{hangparas}{\hpindent}{1}
\large
\Xcomment{Тропарь Покрову Пресвятой Богородицы, глас~4:}
Днесь, благовернии людие, светло празднуем,\,/ осеняеми
Твоим, Богомати, пришествием,\,/ и к Твоему взирающе
пречистому образу, умильно глаголем:\,/ покрый нас честным
Твоим Покровом\,/ и избави нас от всякаго зла,\,/ молящи
Сына Твоего, Христа Бога нашего,\,// спасти души наша.


\Xcomment{Тропарь преподобному Пафнутию Боровскому, глас~4:}
Жития светлостию просветив твое отечество,\,/ в молитвах
и постех дарований Божественнаго Духа исполнился еси,\,/
и, во временней сей жизни добре подвизався,\,/ милость
благоутробия всем скорбящим отверзл еси,\,/ и нищим был еси
заступник.\,/ Тем молим тя, отче Пафнутие,\,// моли Христа
Бога, да спасет души наша.

\Xcomment{Кондак преподобному Пафнутию Боровскому, глас~8:}
Божиим светолитием просвещен, отче,\,/ постническое стяжал
еси жительство, преподобне,\,/ иноком предобрый наставниче
и постником благое украшение.\,/ Сего ради Господь, труды
твоя видев,\,/ чудес даром обогати тя,\,/ источавши бо
исцеления.\,/ Мы же, радующеся, вопием ти:\,// радуйся, отче
Пафнутие.

\Xcomment{Кондак Покрову Пресвятой Богородицы, глас~3:}
Дева днесь предстоит в церкви\,/ и с лики святых невидимо
за ны молится Богу,\,/ Ангели со архиереи покланяются,\,/
апостоли же со пророки ликовствуют:\,// нас бо ради молит
Богородица Превечнаго Бога.

\end{hangparas}

% Во время пения тропарей и кондаков иерей тайно читает молитву Трисвятого пения.
%
% Иерей: Боже, Святый, Иже во святых почиваяй, Иже
% трисвятым гласом от Серафимов воспеваемый, и от Херувимов
% славословимый, и от всякия Небесный Силы покланяемый; Иже
% от небытия во еже быти приведый всяческая, создавый
% человека по образу Твоему и по подобию и всяким Твоим
% дарованием украсйвый; даяй просящему премудрость и разум
% и не презираяй согрешающаго, но полагаяй на спасение
% покаяние; сподобивый нас, смиренных и недостойных раб
% Твоих, и в час сей стати пред славою Святаго Твоего
% Жертвенника и должное Тебе поклонение и славословие
% приносити! Сам, Владыко, приими и от уст нас, грешных,
% Трисвятую песнь и посети ны благостию Твоего, прости нам
% всякое согрешение, вольное же и невольное, освяти наша души
% и телеса и даждь нам в преподобии служити Тебе вся дни
% живота нашего, молитвами Святыя Богородицы и всех святых, от
% века Тебе благоугодивших.

% \diak{Господу помолимся.}
% \xor{Господи, помилуй.}
% \ierey{Яко Свят еси, Боже наш, и Тебе славу возсылаем, Отцу и Сыну и Святому Духу, ныне и присно.}
% \diak{Господи, спаси благочестивыя.}
% \xor{Господи, спаси благочестивыя.}
% \diak{И услыши ны.}
% \xor{И услыши ны.}
% \diak{И во веки веков.}
% \xor{Аминь.}


\ierey{Яко Свят еси, Боже наш, и Тебе славу возсылаем, Отцу и Сыну и Святому Духу, ныне и присно и во веки веков.}
\xor{Аминь.}
\diak{Господи, спаси благочестивыя и услыши ны.}
\xor{Господи, спаси благочестивыя и услыши ны.}


\Subsection{Трисвятое}
\xor{Святый Боже, Святый Крепкий, Святый Безсмертный, помилуй нас. \comment{(Трижды.)}}

\xorX{Слава Отцу и Сыну и Святому Духу, и ныне и присно и во веки веков. Аминь.}

\xorX{Святый Безсмертный, помилуй нас.}

\xorX{Святый Боже, Святый Крепкий, Святый Безсмертный, помилуй нас.}

% На Пасху, в праздники Рождества Христова, Богоявления, в Лазареву и Великую Субботы, во все дни Пасхальной Седмицы, в праздник Пятидесятницы вместо Трисвятого поется:
% Елицы во Христа крестистеся, во Христа облекостеся, аллилуиа.
% В праздник Воздвижения Креста Господня и в Неделю Крестопоклонную поется:
% Кресту Твоему покланяемся, Владыко, и святое Воскресение Твое славим.

\Subsection{Чтение Апостола и Евангелия}
% Диакон: Вонмем.
% Иерей: Мир всем.
% Чтец: И духови твоему.
% Диакон: Премудрость.
% Чтец: Прокимен, глас 3. Песнь Богородицы: Величит душа Моя Господа,\,/ и возрадовася дух Мой о Бозе Спасе Моем. (В Богородичные праздники)
% \xor
% Величит душа Моя Господа,\,/ и возрадовася дух Мой о Бозе Спасе Моем.
% Чтец: Яко призре на смирение Рабы Своея, се бо отныне ублажат Мя вси роди.
% \xor
% Величит дем.
% Чтец: Величит душа Моя Господа.
% \xor
% И возрадовася дух Мой о Бозе Спасе Моем.
%
%
%   Прокимны воскресные на Литургии Глас 1:
% Буди, Господи, милость Твоя на нас,\,/ якоже уповахом на Тя. Глас 2:
% Крепость моя и пение мое Господь,\,/ и бысть мне во спасение. Глас 3:
% Пойте Богу нашему, пойте,\,/ пойте Цареви нашему, пойте. Глас 4:
% Яко возвеличишася дела Твоя, Господи,\,/ вся премудростию
% сотворил еси.
% Ты, Господи, сохраниши ны\,/ и соблюдеши ны от рода сего и
% во век.
% Спаси, Господи, люди Твоя\,/ и благослови достояние Твое.
% Глас 7:
% Господь крепость людем Своим даст,\,/ Господь благословит
% Глас 5:
% Глас 6:
% люди Своя миром.
% Помолитеся и воздадите\,/ Господеви Богу нашему.
% Прокимны дневные (будничные) В понедельник, глас 4:
% Творяй Ангелы Своя духи,\,/ и слуги Своя пламень огненный. Во вторник, глас 7:
% Возвеселится праведник о Господе\,/ и уповает на Него.
% В среду, глас 3:
% Величит душа Моя Господа\,/ и возрадовася дух Мой о Бозе
% Глас 8:
% Спасе Моем.
% Во всю землю изыде вещание их,\,/ и в концы вселенныя
% глаголы их.
% В пятницу, глас 7;
% Возносите Господа Бога нашего,\,/ и покланяйтеся подножию
% ногу Его, яко свято есть.
% В четверг, глас 8:
%
%
%   В субботу, глас 8:
% Веселитеся о Господе\,/ и радуйтеся праведнии. Заупокойный, глас 6:
% Души их\,/ во благих водворятся.
% Диакон: Премудрость,
% Чтец: Деяний святых апостол чтение (или: Соборнаго
% Послания Петрова (или: Иоаннова, причем не принято говорить, какое это послание - первое, второе или третье) чтение; или: К римляном (К коринфяном; К галатом; К Тимофею и т. п.) послания святаго апостола Павла чтение).
% Диакон: Вонмем.
% Чтец читает Апостол.
% Иерей: Мир ти.
% Чтец: И духови твоему.
% Диакон: Премудрость.
\Xcomment{После прочтения Апостола чтец возглашает:}

\chtec{Аллилуиа, аллилуиа, аллилуиа, глас …}
\xor{Аллилуиа, аллилуиа, аллилуиа.}
\chtec{\comment{Произносит стих из псалма.}}
\xor{Аллилуиа, аллилуиа, аллилуиа.}
\chtec{\comment{Произносит еще один стих из псалма.}}
\xor{Аллилуиа, аллилуиа, аллилуиа.}
% Молитва перед чтением Евангелия:
%
% Иерей: Возсияй в сердцах наших, Человеколюбче Владыко,
% Твоего богоразумия нетленный Свет, и мысленный наша
% отверзи очи, во евангельских Твоих проповеданий разумение;
% вложи в нас и страх блаженных Твоих заповедей, да,
% плотскйя похоти вся поправше, духовное жительство пройдем,
% вся, яже ко благоугождению Твоему, и мудрствующе, и деюще.Ты
% бо еси Просвещение душ и телес наших, Христе Боже, и Тебе
% славу возсылаем, со Безначальным Твоим Отцем, и Всесвятым и
% Благим и Животворящим Твоим Духом, ныне и присно и во веки
% веков. Аминь.

% \diak{Благослови, владыко, благовестителя святаго апостола и евангелиста …}
%
% \ierey{Бог, молитвами святаго, славнаго, всехвальнаго апостола и евангелиста … да даст тебе глагол, благо  вествующему силою многою, во исполнение Евангелия Возлюбленнаго Сына Своего, Господа нашего Иисуса Христа.}
%
%
% \diak{Аминь.}

\ierey{Премудрость, прости, услышим Святаго Евангелия. Мир всем.}

\xor{И духови твоему.}

\diak{От … Святаго Евангелия чтение.}

\xor{Слава Тебе, Господи, слава Тебе.}

\ierey{Вонмем.}

\Xcomment{Чтение Евангелия. По  окончании чтения:}
% \ierey{Мир ти, благовествующему. \xor}

\xor{Слава Тебе, Господи, слава Тебе.}

\clearpage
\Subsection{Сугубая ектения}

\diak{Рцем вси от всея души и от всего помышления нашего рцем.}

\xor{Господи, помилуй.}

\diak{Господи Вседержителю, Боже отец наших, молим Ти ся, услыши и помилуй.}

\xor{Господи, помилуй.}

\diak{Помилуй нас, Боже, по велицей милости Твоей, молим Ти ся, услыши и помилуй.}

% \xor{Господи, помилуй. \comment{(Трижды на каждое прошение.)}}
\xor{Господи, помилуй, Господи, помилуй, Господи, помилуй. \comment{(На каждое прошение.)}}

\diak{Еще молимся о Великом Господине и Отце нашем Святейшем Патриарсе Кирилле,
и о Господине нашем Высокопреосвященнейшем митрополите Клименте,
и всей во Христе братии нашей.}

\diakX{Еще
молимся о Богохранимей стране нашей, властех и воинстве ея,
да тихое и безмолвное житие поживем во всяком благочестии и
чистоте.}

\diakX{Еще молимся о братиях наших, священницех, священномонасех и всем во Христе братстве нашем.}

\diakX{Еще молимся о блаженных и приснопамятных святейших
патриарсех православных, и создателех святаго храма сего
% (если в монастыре: святыя обители сея,),
и о всех прежде почивших отцех и братиях, зде лежащих и повсюду, православных.}

\diakX{Еще молимся о милости, жизни, мире, здравии, спасении,
посещении, прощении и оставлении грехов рабов Божиих, братии
святаго храма сего.
% (если в монастыре: святыя обители сея)
}

\diakX{Еще молимся о плодоносящих и добродеющих во святем и всечестнем храме сем,
% (если в монастыре: святой обители сей,),
труждающихся, поющих и предстоящих людех, ожидающих от Тебе великия и богатыя милости.}

% Молитва прилежного моления
% \ierey{Господи, Боже наш, прилежное сие моление приими от
% Твоих раб, и помилуй нас, по множеству милости Твоея, и
% щедроты Твоя низпосли на ны и на вся люди Твоя, чающия от
% Тебе богатыя милости.}

\ierey{Яко милостив и Человеколюбец Бог еси, и Тебе славу возсылаем, Отцу и Сыну и Святому Духу, ныне и присно и во веки веков.}

\xor{Аминь.}

% \Subsection{Заупокойная ектения}
%
% \diak{Помилуй нас, Боже, по велицей милости Твоей, молим Ти ся, услыши и помилуй.}
%
% % \xor{Господи, помилуй. \comment{(Трижды на каждое прошение.)}}
% \xor{Господи, помилуй, Господи, помилуй, Господи, помилуй. \comment{(На каждое прошение.)}}
%
% \diak{Еще молимся о упокоении душ усопших рабов Божиих
% \comment{(имена)} и о еже проститися им всякому прегрешению, вольному
% же и невольному.}
%
% \diakX{Яко да Господь Бог учинит души их, идеже праведнии упокояются.}
%
% \diakX{Милости Божия, Царства Небеснаго и оставления грехов их у Христа, Безсмертнаго Царя и Бога нашего, просим.}
%
% \xor{Подай, Господи.}
% \diak{Господу помолимся.}
%
% \xor{Господи, помилуй.}
%
% \ierey{Боже духов и всякия плоти, смерть поправый и диавола упразднивый, и живот миру Твоему даровавый! Сам, Господи, покой души усопших раб Твоих \comment{(имена)} в месте светле, в месте злачне, в месте покойне, отнюдуже отбеже болезнь, печаль и воздыхание; всякое согрешение, содеянное ими словом, или делом, или помышлением, яко Благий Человеколюбец Бог, прости. Яко несть человек, иже жив будет и не согрешит; Ты бо един токмо без греха, правда Твоя — правда во веки, и слово Твое — истина.}
%
% \ierey{Яко Ты еси Воскресение, и живот, и покой усопших раб Твоих \comment{(имена)}, Христе Боже наш, и Тебе славу возсылаем, со Безначальным Твоим Отцем и Пресвятым и Благим и Животворящим Твоим Духом, ныне и присно и вовеки веков. }
%
% \xor{Аминь.}

% Царские врата закрываются.
\Subsection{Ектения об оглашенных}

\diak{Помолитеся, оглашеннии, Господеви.}

\xor{Господи, помилуй. \comment{(На каждое прошение.)}}

\diak{Вернии, о оглашенных помолимся, да Господь помилует их.}

\diakX{Огласит их словом истины.}

\diakX{Открыет им Евангелие правды.}

\diakX{Соединит их Святей Своей, Соборней и Апостольстей Церкви. }

\diakX{Спаси, помилуй, заступи и сохрани их, Боже, Твоею благодатию.}

\diakX{Оглашеннии, главы ваша Господеви приклоните.}

\xor{Тебе, Господи. \comment{(Протяжно.)}}

% Молитва об оглашенных
%
% \ierey{Господи Боже ваш, Иже на высоких живый и на смиренный призираяй, Иже спасение роду человеческому низпославый — Единороднаго Сына Твоего и Бога, Господа нашего Иисуса Христа. Призри на рабы Твоя оглашенныя, подклоньшия Тебе своя выя, и сподоби я во время благополучное бани пакибытия, оставления грехов и одежди нетления, соедини их Святей Твоей Соборней и Апостольстей Церкви и сопричти их избранному Твоему стаду.}

\ierey{Да и тии с нами славят пречестное и великолепое Имя Твое, Отца и Сына и Святаго Духа, ныне и присно и во веки веков.}

\xor{Аминь.}

\Section{Литургия верных}

\diak{Елицы оглашеннии, изыдите, оглашеннии, изыдите; елицы
оглашеннии, изыдите. Да никто от оглашенных, елицы вернии,
паки и паки миром Господу помолимся.}

\xor{Господи, помилуй. \comment{(Протяжно.)}}

\diak{Заступи, спаси, помилуй и сохрани нас, Боже, Твоею благодатию.}

\xor{Господи, помилуй.}

\diak{Премудрость.}

% Молитва верных первая
%
% \ierey{Благодарим Тя, Господи Боже сил, сподобившего нас
% предстати и ныне Святому Твоему Жертвеннику и припасти ко
% щедротам Твоим о наших гресех и о людских неведениих.
% Приими, Боже, моление наше, сотвори ны достойны быти, еже
% приносити Тебе моления и мольбы и Жертвы Безкровныя о всех
% людех Твоих; и удовли нас, ихже   положил еси в службу
% Твою сию, силою Духа Твоего Святаго, неосужденно и
% непреткновенно, в чистем свидетельстве совести нашея,
% призывати Тя на всякое время и место. Да, послушав нас,
% милостив нам будеши во множестве Твоея благости.}

\ierey{Яко подобает Тебе всякая слава, честь и поклонение, Отцу и Сыну и Святому Духу, ныне и присно и во веки веков.}

\xor{Аминь.}

\diak{Паки и паки миром Господу помолимся.}

\xor{Господи, помилуй. \comment{(Протяжно.)}}
%
% \diak{О свышнем мире и спасении душ наших Господу помолимся.}
%
% \diakX{О мире всего мира, благостоянии святых Божиих Церквей и соединении всех Господу помолимся.}
%
% \diakX{О святем храме сем и с верою, благоговением и страхом Божиим входящих вонь Господу помолимся.}
%
% \diakX{О избавитися нам от всякия скорби, гнева и нужды Господу помолимся.}
%
% \diakX{Заступи, спаси, помилуй и сохрани нас, Боже, Твоею благодатию.}

\diakX{Премудрость.}

% Молитва верных вторая
%
% \ierey{Паки и многажды Тебе припадаем и Тебе молимся, Благий
% и Человеколюбче, яко да, призрев на моление наше, очистиши
% наша души и телеса от всякия скверны плоти и духа и даждь
% нам неповинное и неосужденное предстояние Святаго Твоего
% Жертвенника. Даруй же, Боже, и молящимся с нами преспеяние
% жития, и веры, и разума духовнаго. Даждь им, всегда со
% страхом и любовию служащим Тебе, неповинно и неосужденно
% причаститися Святых Твоих Тайн, и Небеснаго Твоего Царствия
% сподобитися.}

\ierey{Яко да под державою Твоею всегда храними, Тебе славу
возсылаем, Отцу и Сыну и Святому Духу, ныне и присно и во
веки веков.}

\xor{Аминь. \comment{(Протяжно.)}}
% Отверзаются Царские врата.

\clearpage
\Subsection{Херувимская песнь}
\xor{
Иже Херувимы тайно образующе и Животворящей
Троице Трисвятую песнь припевающе, всякое ныне житейское
отложим попечение.}

% Во время пения Херувимской песни диакон, взяв благословение от иерея, Читая про себя 50-й псалом, кадит Святой
% Престол, Святой Жертвенник, алтарь, затем Царскими вратами выходит на солею и кадит иконостас: входит в алтарь, кадит иерея и предстоящих там; опять выходит на ам¬вон и кадит лики и молящихся. Иерей во время каждения читает тайно молитву Херувимской песни.
% \ierey{Никтоже достоин от связавшихся плотскими похотьми и сластьми приходити, или приближитися, или служити Тебе, Царю славы; еже бо служити Тебе — велико и страшно и самем Небесным Силам. Но обаче, неизреченнаго ради и безмернаго Твоего человеколюбия, непреложно и неизменно был еси Человек и Архиерей нам был еси, и служебный сея и Безкровныя Жертвы священнодействие предал еси нам, яко Владыка всех. Ты бо един, Господи Боже наш, владычествуе-ши Небесными и земными, Иже на Престоле Херувимсте носимый, Иже Серафимов Господь и Царь Израилев, Иже един Свят и во святых почиваяй. Тя убо молю, единаго Благаго и Благопослушливаго: призри на мя, грешнаго и непотребнаго раба Твоего, и очисти мою душу и сердце от совести лукавыя, и удовли мя, силою Святаго Твоего Духа, облеченна благодатию священства, предстати святей Твоей сей трапезе и священнодействовати Святое и Пречистое Твое Тело и Честную Кровь. К Тебе бо прихожду, приклонь мою выю, и молю Ти ся, да ие отвратиши лица Твоего от мене, ниже отринеши мене от отрок Твоих. Но сподоби принесѐнным Тебе быти много, грешным и недостойным рабом Твоим, Даром сим. Ты бо еси приносяй и приносимый, и приемляй и раздаваемый, Христе Боже наш, и Тебе славу возсылаем, со Безначальным Твоим Отцем и Пресвятым и Благим и Животворящим Твоим Духом, ныне и присно и во веки веков. Аминь.
% По окончании чтения иереем этой молитвы, а диаконом — каждения, оба они с умилением трижды вполголоса читают Херувимскую песнь (на Литургии св. Василия Великого в Великий Четверг читают: «Вечери Твоея Тайныя...», а в Великую Субботу - «Да молчит всякая плоть человеча...» ), затем отходят к Святому Жертвеннику.
% Иерей кадит Святое Предложение, молясь тайно: Боже, очисти мя, грешнаго, (Трижды )
% \diak{Возьми, владыко.}
%
% Иерей, возлагая на его плечо воздух: Возьмите руки ваша во Святая и благословите Господа.
\Xcomment{Иерей и диакон исходят из алтаря северными вратами на солею.}

 \diak{Великаго Господина и Отца нашего Кирилла, Святейшего Патриарха Московскаго и всея Руси
и Господина нашего Высокопреосвященнейшего Климента, митрополита Калужскаго и Боровскаго,
да помянет Господь Бог во Царствии Своем всегда, ныне и присно и во веки веков.}

\iereyX{Преосвященныя митрополиты, архиепископы и епископы, и весь священнический и монашеский чин, причет церковный, благотворителей, жертвователей святаго храма сего. Вас и всех православных христиан да помянет Господь Бог во Царствии Своем всегда, ныне и присно и во веки веков.}

\xor{Аминь. Яко да Царя всех подымем, ангельскими невидимо дориносима чинми. Аллилуиа, аллилуиа, аллилуиа.}
% Иерей и диакон входят, в алтарь.
% \diak{Да помянет Господь Бог священство твое во Царствии Своем.}
%
% \ierey{Да помянет Господь Бог священнодиаконство твое во Царствии Своем, всегда, ныне и присно и во веки веков.}
%
% Иерей, поставляя Дары на Престол, читает тропари: «Благообразный Иосиф...», «Во гробе плотски, во аде же с душею, яко Бог...», «Яко Живоносец, яко Рая краснейший...» Ублажи, Господи, благоволением Твоим Сиона, и да созиждутся стены Иерусалимския; тогда благоволиши жертву правды, возношение н всесожегаемая, тогда возложат на олтарь Твой тельцы.
% Иерей диакону: Помолися о мне, брате и сослужителю.
% \diak{Дух Святый найдет на тя, н сила Вышняго осенит тя.}
%
% \ierey{Тойже Дух содействует нам вся дни живота нашего.}
%
% \diak{Помяни и мя, владыко святый.}
%
% Иерей, благословляя: Да помянет тя Господь Бог во Царствии Своем всегда, ныне и присно и во веки веков.
% \diak{Аминь.}
%
% Закрываются Царские врата и завеса.
%

\clearpage
\Subsection{Просительная ектения}
\diak{Исполним молитву нашу Господеви.}

\xor{Господи, помилуй. \comment{(На каждое прошение.)}}

\diak{О предложенных честных Дарех Господу помолимся.}

\diakX{О святем храме сем и с верою, благоговением и страхом Божиим входящих в онь Господу помолимся.}

\diakX{О избавитися нам от всякия скорби, гнева и нужды Господу помолимся.}

\diakX{Заступи, спаси, помилуй и сохрани нас, Боже, Твоею благодатию.}

\diakX{Дне всего совершенна, свята, мирна и безгрешна у Господа просим.}

\xor{Подай, Господи. \comment{(На каждое прошение.)}}

\diak{Ангела мирна, верна наставника, хранителя душ и телес наших у Господа просим.}

\diakX{Прощения и оставления грехов и прегрешений наших у Господа просим.}

\diakX{Добрых и полезных душам нашим и мира мирови у Господа просим.}

\diakX{Прочее время живота нашего в мире и покаянии скончати у Господа просим.}

\diakX{Христианския кончины живота нашего, безболезнены, непостыдны, мирны, и добраго ответа на Страшнем Судищи Христове просим.}

\diakX{Пресвятую, Пречистую, Преблагословенную, Славную Владычицу нашу Богородицу и Приснодеву Марию, со всеми святыми помянувше, сами себе, и друг друга, и весь живот наш Христу Богу предадим.}

\xor{Тебе, Господи.}

%     Иерей: Господи Боже Вседержителю, Едине Святе,
% приемляй жертву хваления от призывающих Тя всем сердцем!
% Приими и нас, грешных, моление, и принеси ко Святому Твоему
% Жертвеннику, и удовли нас приносити Тебе дары же и жертвы
% духовныя о наших гресех и о людских неведениих; и сподоби
% нас обрести благодать пред Тобою, еже быти Тебе
% благоприятней жертве нашей и вселитися Духу благодати Твоея
% Благому в нас, и на предлежащих Дарех сих, и на всех людех
% Твоих.

\ierey{Щедротами Единороднаго Сына Твоего, с Нимже
благословен еси, со Пресвятым и Благим и Животворящим Твоим
Духом, ныне и присно и во веки веков.}

\xor{Аминь.}

\ierey{Мир всем.}

\xor{И духови твоему.}
\diak{Возлюбим друг друга, да единомыслием исповемы.}

\xor{Отца и Сына и Святаго Духа, Троицу Единосущную и
Нераздельную.}

% Иерей, поклоняясь: Возлюблю Тя, Господи, крепосте моя, Господь утверждение мое и прибежище мое. (Трижды)
\diak{Двери, двери, премудростию вонмем. }

% Открывается завеса Царских врат.
\clearpage

\Subsection{Символ веры} \xor{Верую во Единаго Бога Отца
Вседержителя, Творца небу и земли, видимым же всем и
невидимым. И во Единаго Господа Иисуса Христа, Сына Божия,
Единороднаго, Иже от Отца рожденнаго прежде всех век. Света
от Света, Бога истинна от Бога истинна, рожденна,
несотворенна, Единосущна Отцу, Имже вся быша. Нас ради
человек и нашего ради спасения сшедшаго с Небес, и
воплотившагося от Духа Свята и Марии Девы, и вочеловечшася.
Распятаго же за ны при Понтийстем Пилате, и страдавша, и
погребенна. И   воскресшаго в третий день по Писанием.
И восшедшаго на Небеса, и седяща одесную Отца. И паки
грядущаго со славою судити живым и \noh{мертвым,} Егоже Царствию
не будет конца. И в Духа Святаго, Господа Животворящаго, Иже
от Отца исходящаго, Иже со Отцем и Сыном спокланяема и
сславима, глаголавшаго пророки. Во едину Святую Соборную и
Апостольскую Церковь. Исповедую едино Крещение во оставление
грехов. Чаю воскресения мертвых и жизни будущаго века.
Аминь.}

\diak{Станем добре, станем со страхом, вонмем, Святое Возношение в мире приносити.}

\xor{Милость мира, жертву хваления.}

\ierey{Благодать Господа нашего Иисуса Христа, и любы Бога и Отца, и причастие Святаго Духа буди со всеми вами.}

\xor{И со духом твоим.}

\ierey{Горе имеим сердца.}

\xor{Имамы ко Господу.}

\ierey{Благодарим Господа.}

\clearpage
\xor{Достойно и праведно есть покланятися Отцу и Сыну и
Святому Духу, Троице Единосущней и Нераздельней.}

% \ierey{Достойно и праведно Тя пети, Тя благословити, Тя
% хвалити, Тя благодарити, Тебе покланятися на всяком месте
% владычествия Твоего. Ты бо еси Бог Неизречен, Недоведом,
% Невидим, Непостижим, присно Сый, такожде Сый, Ты, и
% Единородный Твой   Сын, и Дух Твой Святый. Ты от
% небытия в бытие нас привел еси, и отпадшия возставил еси
% паки, и не отступил еси, вся творя, дондеже нас на Небо
% возвел еси и Царство Твое даровал еси будущее. О сих всех
% благодарим Тя, и Единороднаго Твоего Сына, и Духа Твоего
% Святаго о всех, ихже вемы и ихже не вемы, явленных и
% неявленных благодеяниих, бывших на нас. Благодарим Тя и о
% службе сей, юже от рук наших прияти изволил еси, аще и
% предстоят Тебе тысящи Архангелов и тьмы Ангелов, Херувими и
% Серафими, шестокрилатии, многоочитии, возвышающиеся,
% пернатии.}

\ierey{Победную песнь поюще, вопиюще, взывающе и глаголюще.}

\xor{Свят, Свят, Свят Господь Саваоф, исполнь Небо и земля
славы Твоея; осанна в вышних, благословен Грядый во Имя
Господне, осанна в вышних.}

% \ierey{С сими и мы блаженными Силами, Владыко Человеколюбче,
% вопием и глаголем: Свят еси и Пресвят, Ты, и Единородный
% Твой Сын, и Дух Твой Святый; Свят еси и Пресвят, и
% великолепна слава Твоя; Иже мир Твой тако возлюбил еси,
% якоже Сына Своего Единороднаго дати, да всяк веруяй в Него
% не погибнет, но имать живот вечный. Иже, пришед и все еже о
% нас смотрение исполнив, в нощь, в нюже предаяшеся, паче же
% Сам Себе предаяшее за мирский живот, прием хлеб во Святыя
% Своя и Пречистыя и непорочныя руки, благодарив и
% благословив, освятив, преломив, даде святым Своим учеником и
% апостолом, рек:}

\ierey{Приимите, ядите, Сие есть Тело Мое, еже за вы ломимое во оставление грехов.}
\printsmall{На Литургии свт. Василия Великого иерей: }{Даде святым Своим учеником и
апостолом, рек: Приимите, ядите, Сие есть Тело Мое, еже за вы ломимое во оставление грехов.}

\xor{Аминь.}

% \ierey{Подобие и Чашу по вечери, глаголя:}

\ierey{Пийте от нея вси, Сия есть Кровь Моя Новаго Завета, яже за вы и за многи изливаемая во оставление грехов.}
\printsmall{На Литургии свт. Василия Великого иерей: }{Даде святым Своим учеником и
апостолом, рек: Пийте от нея вси, Сия есть Кровь Моя Новаго Завета, яже за вы и за многи изливаемая во оставление грехов.}

\xor{Аминь.}

% \ierey{Поминающе убо спасительную сию заповедь и вся, яже о
% нас бышая: Крест, Гроб, тридневное Воскресение, на Небеса
% восхождение, одесную седение, второе и славное паки
% пришествие.}

\ierey{Твоя от Твоих Тебе приносяще о всех и за вся.}

\xor{Тебе поем, Тебе благословим, Тебе благодарим,
Господи, и молим Ти ся, Боже Наш.}

% \ierey{Еще приносим Ти словесную сию и безкровную службу, и
% просим, и молим, и мили ся деем, низпосли Духа Твоего
% Святаго на ны и на предлежащия Дары сия. Господи, Иже
% Пресвятаго Твоего Духа в третий час Апостолом Твоим
% низпославый, Того, Благий, не отыми от нас, но обнови нас,
% молящих Ти ся. (Трижды, поклоняясь перед престолом)}
%
% * На Литургии св. Василия Великого эти возгласы предваряются словами:
% Дадѐ святым Своим учеником и апостолом, рек.
% Диакон (после первого произнесения тропаря): Сердце чисто созижди во мне, Боже, и дух прав обнови во утробе моей.
% Диакон (после второго произнесения тропаря): Не отвержи мене от лица Твоего и Духа Твоего Святаго не отымй от мене.
% Диакон (указывая орарем на Святой Хлеб): Благослови, владыко, Святый Хлеб.
% Иерей (благословляя): И сотвори убо Хлеб Сей Честное Тело Христа Твоего.
% \diak{Аминь. Благослови, владыко, Святую Чашу.}
%
% Иерей (благословляя): А еже в Чаши сей, Честную Кровь Христа Твоего.
% \diak{Аминь. Благослови, владыко, обоя (указывая орарем на Святые Дары).}
%
% Иерей (благословляя Святые Дары вместе): Преложив Духом Твоим Святым.
% \diak{Аминь, аминь, аминь.}
%
% \diak{Помяни мя, святый владыко, грешнаго.}
%
% \ierey{Да помянет тя Господь Бог во Царствии Своем всегда, ныне и присно и во веки веков.}
%
% \diak{Аминь.}
%
% \ierey{Якоже быти причащающимся, во трезвение души, во оставление грехов, в приобщение Святаго Твоего Духа, во исполнение Царствия Небеснаго, в дерзновение еже к Тебе, не в суд или во осуждение. Еще приносим Ти словесную сию службу о иже в вере почивших праотцех, отцех, патриарсех, пророцех, апостолех, проповедницех, евангелистех, мученицех, исповедницех, воздержницех и о всяком дусе праведней, в вере скончавшемся.}

\ierey{Изрядно о Пресвятей, Пречистей, Преблагословенней,
Славней Владычице нашей Богородице и Приснодеве Марии.}

 \xor{Достойно есть, яко воистину блажити Тя, Богородицу,
Присноблаженную и Пренепорочную и Матерь Бога нашего.
Честнейшую Херувим и Славнейшую без сравнения Серафим, без
изстления Бога Слова рождшую, сущую Богородицу Тя величаем.}

\printlarge{На Литургии свт. Василия Великого вместо «Достойно есть…» мы: }%
{О~Тебе радуется, Благодатная, всякая тварь, ангельский собор и человеческий род, освященный Храме и Раю словесный, девственная похвало, из Неяже Бог воплотися и младенец бысть, прежде век сый Бог наш; ложесна бо Твоя престол сотвори и чрево Твое пространнее небес содела. О Тебе радуется, Благодатная, всякая тварь, слава Тебе.}

% \ierey{О святем Иоанне Пророце, Предтечи и Крестители, о святых славных и всехвальных Апостолех, о святем (имя), егоже и память совершаем, и о всех святых Твоих, ихже молитвами посети нас, Боже. И помяни всех усопших о надежди воскресения жизни вечныя (имена).И упокой их, идеже присещает свет лица Твоего, Еще молим Тя, помяни, Господи, всякое епископство православных, право правящих слово Твоея истины, всякое пресвитерство, во Христе диаконство и всякий священнический чин. Еще приносим Ти словесную сию службу о вселенней, о Святей, Соборней и Апостольстей Церкви, о иже в чистоте и честнем жительстве пребывающих; о Богохранимей стране нашей, властех и воинстве ея. Даждь им, Господи, мирное правление, да и мы в тишине их тихое и безмолвное житие поживем, во всяком благочестии и чистоте.}

\ierey{В первых помяни, Господи, Великаго Господина и Отца нашего Кирилла, Святейшаго Патриарха Московскаго и всея Руси,
и Господина нашего Высокопреосвященнейшего Климента, митрополита Калужскаго и Боровскаго,
ихже даруй святым Твоим Церквам в мире, целых, честных, здравых, долгоденствующих, право правящих слово Твоея истины.}
% и Господина нашего Преосвященнейшаго (имя епархиального архиерея), ихже даруй святым Твоим Церквам в мире, целых, честных, здравых, долгоденствующих, право правящих слово Твоея истины.


\xor{И всех и вся.}

% \ierey{Помяни, Господи, град сей, в немже живем (или: весь сию, в нейже живем, или:
% обитель сию), и всякий град и страну и верою живущих в них. Помяни, Господи,
%   плавающих, путешествующих, недугующих, страждущих, плененных, и спасение их. Помяни, Господи, плодоносящих и добротворящих во святых Твоих церквах, и поминающих убогия, и на вся ны милости Твоя ни зпосли (имена живых).

\ierey{И даждь нам едиными усты и единым сердцем славити и
воспевати Пречестное и Великолепое Имя Твое, Отца и Сына и
Святаго Духа, ныне и присно и во веки веков.}

\xor{Аминь.}
\ierey{И да будут милости Великаго Бога и Спаса нашего
Иисуса Христа со всеми вами. }

\xor{И со духом твоим.}

\clearpage
\Subsection{Просительная ектения}

\diak{Вся святыя помянувше, паки и паки миром Господу помолимся.}

\xor{Господи, помилуй. \comment{(На каждое прошение.)}}

\diak{О принесенных и освященных Честных Дарех Господу помолимся.}

\diakX{Яко да Человеколюбец Бог наш, приемь я во святый, и пренебесный, и мысленный Свой Жертвенник, в воню благоухания духовнаго, возниспослет нам Божественную благодать и дар Святаго Духа, помолимся.}

\diakX{О избавитися нам от всякия скорби, гнева и нужды Господу помолимся.}

\diakX{Заступи, спаси, помилуй и сохрани нас, Боже, Твоею благодатию.}

\diakX{Дне всего совершенна, свята, мирна и безгрешна у Господа просим.}

  \xor{Подай, Господи. \comment{(На каждое прошение.)}}

\diak{Ангела мирна, верна наставника, хранителя душ и телес наших у Господа просим.}

\diakX{Прощения и оставления грехов и прегрешений наших у Господа просим.}

\diakX{Добрых и полезных душам нашим и мира мирови у Господа просим.}

\diakX{Прочее время живота нашего в мире и покаянии скончати у Господа просим.}

\diakX{Христианския кончины живота нашего, безболезнены,
непостыдны, мирны, и добраго ответа на Страшнем Судищи
Христове просим.}

\diakX{Соединение веры и причастие Святаго Духа испросивше, сами себе, и друг друга, и весь живот наш Христу Богу предадим.}

\xor{Тебе, Господи.}

% \ierey{Тебе предлагаем живот наш весь и надежду, Владыко
% Человеколюбче, и просим, и молим, и мили ся деем: сподоби
% нас причаститися Небесных Твоих и Страшных Тайн, сея
% священныя и духовный Трапезы, с чистою совестию, во
% оставление грехов, в прощение согрешений, во общение Духа
% Святаго, в наследие Царствия Небеснаго, в дерзновение еже к
% Тебе, не в суд или во осуждение.}

\ierey{И сподоби нас, Владыко, со дерзновением, неосужденно смети призывати Тебе, Небеснаго Бога Отца, и глаголати:}

\xor{Отче наш, Иже еси на Небесех! Да святится Имя Твое, да приидет Царствие Твое, да будет воля Твоя, яко на Небеси и на земли. Хлеб наш насущный даждь нам днесь, и остави нам долги наша, якоже и мы оставляем должником нашим; и не введи нас во искушение, но избави нас от лукаваго.}

 \ierey{Яко Твое есть Царство, и сила, и слава, Отца и Сына и Святаго Духа, ныне и присно и во веки веков.}
\xor{Аминь.}
\ierey{Мир всем.}

\xor{И духови твоему.}
\diak{Главы ваша Господеви приклоните.}

\xor{Тебе, Господи. \comment{(Протяжно.)}}

% \ierey{Благодарим Тя, Царю невидимый, Иже неисчетною Твоею
% силою вся содетельствовал еси и множеством милости Твоея от
% небытия в бытие вся привел еси. Сам, Владыко, с Небесе
% призри на подклоньшия Тебе главы своя; не бо подклониша
% плоти и крови, но Тебе, Страшному Богу. Ты убо, Владыко,
% предлежащая всем нам во благое изравняй, по коегождо своей
% потребе: плавающим сплавай, путешествующим спутешествуй,
% недугующия исцели, Врачу душ и телес.}

\ierey{Благодатию, и щедротами, и человеколюбием
Единороднаго Сына Твоего, с Нимже благословен еси, со
Пресвятым и Благим и Животворящим Твоим Духом, ныне и присно
и во веки веков.}

\xor{Аминь. \comment{(Протяжно.)}}

% \ierey{Вонми, Господи Иисусе Христе, Боже наш, от святаго
% жилища Твоего и от Престола славы Царствия Твоего, и прииди
% во еже освятити нас, Иже горе со Отцем седяй и зде нам
% невидимо спребываяй, и сподоби державною Твоею рукою
% преподати нам Пречистое Тело Твое и Честную Кровь, и нами —
% всем людем.}

\diak{Вонмем.}

%   Закрывается завеса Царских врат.

\ierey{Святая святым.}

\xor{Един Свят, един Господь Иисус Христос, во славу Бога Отца. Аминь.}

\Xcomment{И поется причастный стих. В воскресеный день:}
\xor{Хвалите Господа с небес, хвалите Его в вышних. Аллилуиа, аллилуиа, аллилуиа.}
%
% Прйчастоц
% В понедельник:
% Творяй Ангелы Своя духи, и слуги Своя пламень огненный.
% Во вторник:
% В память вечную будет праведник, от слуха зла не убоится.
% В среду:
% Чашу спасения прииму и Имя Господне призову.
% В четверг:
% Во всю землю изыде вещание их, и в концы вселенныя
% глаголы их.
% В пятницу:
% Спасение соделал еси посреде земли, Боже.
% В субботу:
% Радуйтеся, праведнии, о Господе, правым подобает похвала.
% Заупокойный:
% Блажени, яже избрал и приял еси, Господи, и память их в род
% и род.
% В праздники Богородичные:
% Чашу спасения прииму и Имя Господне призову.
% В дни памяти святых:
% В память вечную будет праведник, от слуха зла не убоится.
%
%
%   В праздники Апостолов:
% Во всю землю изыде вещание их, и I в концы вселенныя глаголы их.
% В это время происходит Причащение священнослужителей в алтаре.
%
% После этого
% \Xcomment{И псалом 33:}
%
% \xor{Благословлю Господа на всякое время, выну хвала Его во
% устех моих. О Господе похвалится душа моя, да услышат
% кротции, и возвеселятся. Возвеличите Господа со мною, и
% вознесем Имя Его вкупе. Взысках Господа, и услыша мя, и от
% всех скорбей моих избави мя. Приступите к Нему и
% просветитеся, и лица ваша не постыдятся. Сей нищий воззва,
% и Господь услыша и, и от всех скорбей его спасе и.
% Ополчится ангел Господень окрест боящихся Его, и избавит
% их. Вкусите, и видите, яко благ Господь: блажен муж, иже
% уповает Нань. Бойтеся Господа вси святии Его, яко несть
% лишения боящимся Его. Богатии обнищаша и взалкаша:
% взыскающии же Господа не лишатся всякаго блага. Приидите,
% чада, послушайте мене, страху Господню научу вас. Кто есть
% человек хотяй живот, любяй дни видети благи? Удержи язык
% твой от зла, и устне твои, еже не глаголати льсти. Уклонися
% от зла и сотвори благо. Взыщи мира, и пожени и. Очи Господни
% на \mbox{праведныя,} и уши Его в молитву их. Лице же Господне на
% творящыя злая, еже потребити от земли память их. Воззваша
% праведнии, и Господь услыша их, и от всех скорбей их избави
% их. Близ Господь сокрушенных сердцем, и смиренныя духом
% спасет. Многи скорби праведным, и от всех их избавит я
% Господь. Хранит Господь вся кости их, ни едина от них
% сокрушится. Смерть грешников люта, и ненавидящии праведнаго
% прегрешат. Избавит Господь душы раб Своих, и не прегрешат
% вси уповающии на Него. }

\Xcomment{Отверзаются Царские врата.}

 % Диакон, выйдя вместе с иереем на солею через Царские врата и подняв Святую Чашу вверх, произносит:

\diak{Со страхом Божиим и верою приступите.}

\xor{Благословен Грядый во Имя Господне, Бог Господь и
явися нам.}

\ierey{Верую, Господи, и исповедую, яко Ты еси воистину
Христос, Сын Бога Живаго, пришедый в мир грешныя спасти, от
нихже первый есмь аз. Еще верую, яко Сие самое есть
Пречистое Тело Твое, и Сия самая есть Честная Кровь Твоя.
Молюся убо Тебе: помилуй мя, и прости ми прегрешения моя,
вольная и невольная, яже словом, яже делом, яже ведением и
неведением, и сподоби мя неосужденно причаститися Пречистых
Твоих Таинств, во оставление грехов и в жизнь вечную. Аминь.
Вечери Твоея Тайныя днесь, Сыне Божий, причастника мя
приими; не бо врагом Твоим тайну повем, ни лобзания ти дам
яко Иуда, но яко разбойник исповедаю Тя: помяни мя, Господи,
во Царствии Твоем. Да не в суд или во осуждение будет мне
причащение Святых Твоих Тайн, Господи, но во исцеление души
и тела.}

% Причащая мирян, иерей глаголет: Причащается раб Божий (имя) Честнаго и Святаго Тела и Крове Господа и Бога и Спаса
%   нашего Иисуса Христа, во оставление грехов своих и в жизнь вечную.

\Xcomment{Во время причащения:}

\xor{Тело Христово приимите, Источника безсмертнаго вкусите.}

\Xcomment{По окончании причащения:}

\xor{Аллилуиа, аллилуиа, аллилуиа.}
%
% * В пасхальную Седмицу поется «Христос воскресе...»
% ** При причащении младенцев иерей произносит: «Честныя и Святы я Крове Господа и Бога и Спаса нашего Иисуса Христа причащается младенец (имя) в жизнь вечную».
% *** В Великий Четверг поется «Вечери Твоея тайныя...», в пасхальную Седмицу поется «Христос воскресе... »
% После причащения мирян иерей с диаконом входят, в алтарь. Иерей ставит Святую Чашу на Святой Престол, а диакон, взяв Святой Дискос, читает воскресные (пасхальные) песнопения: «Воскресение Христово видевше..,», «Светися, светися...», «О Пасха велия и священнейшая, Христе..»
% Затем диакон опускает в Потир частицы, вынутые из просфор, лежащих на Дискосе, со словами: Отмый, Господи, грехи поминавшихся зде Кровию Твоею Честною, молитвами святых Твоих.

\ierey{Спаси, Боже, люди Твоя и благослови достояние Твое.}

% Во время пения «Видехом свет истинный...» иерей кадит трижды Святые Дары, говоря: Иерей: Вознесйся на Небеса, Боже, и по всей земли слава Твоя.
\xor{Видехом Свет Истинный, прияхом Духа Небеснаго, обретохом веру истинную, Нераздельней Троице покланяемся, Та бо нас спасла есть.}
% \ierey{Благословен Бог наш.}
% Иерей (обратившись лицом к народу со Святой Чашей):

\ierey{Всегда, ныне и присно и во веки веков.}


\xor{Аминь. Да исполнятся уста наша хваления Твоего, Господи, яко да поем славу Твою, яко сподобил еси нас причаститися Святым Твоим, Божественным, Безсмертным и Животоворящим Тайнам; соблюди нас во Твоей святыни, весь день поучатися правде Твоей. Аллилуиа, аллилуиа, аллилуиа.}

\diak{Прости приимше Божественных, Святых, Пречистых, Безсмертных, Небесных и Животворящих, Страшных Христовых Тайн, достойно благодарим Господа.}

\xor{Господи, помилуй. \comment{(На каждое прошение.)}}

\diak{Заступи, спаси, помилуй и сохрани нас, Боже, Твоею благодатию.}

\diakX{День весь совершен, свят, мирен и безгрешен испросивше, сами себе, и друг друга, и весь живот наш Христу Богу предадим.}
\xor{Тебе, Господи.}

% Иерей, сложив Антиминс и держа вертикально Святое Евангелие, делает им знак креста
% над Антиминсом и произносит:
\ierey{Яко Ты еси Освящение наше, и Тебе славу возсылаем, Отцу и Сыну и Святому Духу, ныне и присно и во веки веков.}

\xor{Аминь.}
\ierey{С миром изыдем.}

\xor{О имени Господни.}

\diak{Господу помолимся.}
\xor{Господи, помилуй.}
% Мпмйтфа чаанфпооая

\ierey{Благословляяй благословящия Тя, Господи, и освящаяй
на Тя уповающия, спаси люди Твоя и благослови достояние
Твое, исполнение Церкве Твоея сохрани, освяти любящия
благолепие дому Твоего; Ты тех возпрослави Божественною
Твоею силою, и не остави нас, уповающих на Тя. Мир мирови
Твоему даруй, Церквам Твоим, священником, воинству и всем
людем Твоим. Яко всякое даяние благо, и всяк дар совершен
свыше есть, сходяй от Тебе, Отца Светов; и Тебе славу, и
благодарение, и поклонение возсылаем, Отцу и Сыну и Святому
Духу, ныне и присно и во веки веков.}

\xor{Аминь.}
\xor{Буди Имя Господне благословенно отныне и до века. \comment{(Трижды.)}}
% Псампн 33**
% \xor
% Благословлю Господа на всякое время,\,/ выну хвала Его во устех моих.\,/ О Господе похвалится душа моя,\,/ да услышат кротции, и возвеселятся.\,/ Возвеличите Господа со мною,\,/ и вознесѐм Имя Его вкупе.\,/ Взысках Господа, и услыша мя,\,/ и от всех скорбей моих избави мя. /Приступите к Нему и просветитеся,\,/ и лица ваша не постыдятся.\,/ Сей нищий воззва, и Господь услыша и,\,/ и от всех скорбей его спасе и.\,/ Ополчится ангел Господень окрест боящихся Его,\,/ и избавит   их. /Вкусите, и видите, яко благ Господь:\,/ блажен муж, иже уповает Нань. /Бойтеся Господа вси святии Его,\,/ яко несть лишения боящимся Его. /Богатии обнищаша и взалкаша:\,/ взыскающии же Господа //не лишатся всякаго блага.
% * Па пасхальной Седмице поется «Христос воскресе... »
% **На пасхальной Седмице вместо 33-го псалма поется многократно «Христос воскресе»

\ierey{Благословение Господне на вас, Того благодатию и человеколюбием, всегда, ныне и присно и во веки веков.}

\xor{Аминь.}
\ierey{Слава Тебе, Христе Боже, Упование наше, слава Тебе.}

\xor{Слава Отцу и Сыну и Святому Духу, и ныне и присно и во веки веков, аминь.}
% \xor{Господи, помилуй. \comment{(Трижды.)}}
\xorX{Господи, помилуй, Господи, помилуй, Господи, помилуй.}
\xorX{Благослови.}

\ierey{\comment{Воскресый из мертвых,} Христос, истинный Бог наш, молитвами Пречистыя Своея Матере, святых славных и всехвальных Апостол, иже во святых отца нашего Иоанна, архиепископа Константина града, Златоустаго, \comment{(святого, которого память в этот день),} святых и праведных Богоотец Иоакима и Анны и всех святых, помилует и спасет нас, яко Благ и Человеколюбец.}

\xor{Великаго Господина и Отца нашего Кирилла, Святейшаго Патриарха Московскаго и всея Руси,
и Господина нашего Высокопреосвященнейшего Климента, митрополита Калужскаго и Боровскаго,
братию святаго храма сего и вся православныя христианы, Господи, сохрани их на многая лета.}

\clearpage
\pagestyle{empty}
\vspace*{\fill}
{\centering КОНЕЦ\\
И БОГУ НАШЕМУ \\
СЛАВА\par}
\vspace*{\fill}
\clearpage
% \
% \clearpage
\vspace*{\fill}
\fontsize{8}{10}\selectfont
\centering
\fontspec{PTM55F.ttf}

Свёрстано для пения на Божественной Литургии
в~воскресный день
при~служении иерея без~диакона\\
на подворье Свято-Пафнутьева Боровского монастыря\\
в храм Покрова Пресвятой Богородицы в г. Боровске.

\vspace{6pt}
Подготовлено и отпечатано в издательстве «Орфограф»\\
www.orfograf.com\\
info@orfograf.com\\
+7 (926) 449$\cdot$61$\cdot$99
% Исходный код в формате \LaTeX{} доступен по адресу\\
% https://github.com/pantlmn/liturgiya-s-narodom


\vspace{6pt}
Тираж 300 экз

\vspace{6pt}
Москва, 2018 год
